% \iffalse
%<*driver>
\ProvidesFile{\jobname.dtx}[2020/07/11 v0.3.2-dev An Unofficial Thesis Template for ShanghaiTech University]
\documentclass{ltxdoc}
\usepackage{\jobname-doc}

\EnableCrossrefs
\CodelineIndex
\RecordChanges

\begin{document}
  \DocInput{\jobname.dtx}
\end{document}
%</driver>
% \fi
%
% ^^A DoNotIndex list from thuthesis
% ^^A https://github.com/tuna/thuthesis/blob/17acabe/thuthesis.dtx#L33-L52
% \DoNotIndex{\newenvironment,\@bsphack,\@empty,\@esphack,\sfcode}
% \DoNotIndex{\addtocounter,\label,\let,\linewidth,\newcounter}
% \DoNotIndex{\noindent,\normalfont,\par,\parskip,\phantomsection}
% \DoNotIndex{\providecommand,\ProvidesPackage,\refstepcounter}
% \DoNotIndex{\RequirePackage,\setcounter,\setlength,\string,\strut}
% \DoNotIndex{\textbackslash,\texttt,\ttfamily,\usepackage}
% \DoNotIndex{\begin,\end,\begingroup,\endgroup,\par,\\}
% \DoNotIndex{\if,\ifx,\ifdim,\ifnum,\ifcase,\else,\or,\fi}
% \DoNotIndex{\let,\def,\xdef,\edef,\newcommand,\renewcommand}
% \DoNotIndex{\expandafter,\csname,\endcsname,\relax,\protect}
% \DoNotIndex{\Huge,\huge,\LARGE,\Large,\large,\normalsize}
% \DoNotIndex{\small,\footnotesize,\scriptsize,\tiny}
% \DoNotIndex{\normalfont,\bfseries,\slshape,\sffamily,\interlinepenalty}
% \DoNotIndex{\textbf,\textit,\textsf,\textsc,\textup}
% \DoNotIndex{\hfil,\par,\hskip,\vskip,\vspace,\quad}
% \DoNotIndex{\centering,\raggedright,\ref}
% \DoNotIndex{\c@secnumdepth,\@startsection,\@setfontsize}
% \DoNotIndex{\ ,\@plus,\@minus,\p@,\z@,\@m,\@M,\@ne,\m@ne}
% \DoNotIndex{\@@par,\DeclareOperation,\RequirePackage,\LoadClass}
% \DoNotIndex{\AtBeginDocument,\AtEndDocument}
% \DoNotIndex{\NeedsTeXFormat,\ProvidesClass}
%
% \GetFileInfo{\jobname.dtx}
%
% \title{\ShtThesis{} \fileversion{} 用户指南}
% \author{李润东\thanks{\href{mailto:rundong.001@gmail.com}{rundong.001@gmail.com}}}
%
% \changes{v0.3.2-dev}{2020/07/12}{使用 \textsf{Doc} 和 \textsf{DocStrip} 重构项目}
%
% \maketitle
% 
% \begin{abstract}
% \shtthesis{} (\textbf{S}hang\textbf{h}ai\textbf{T}ech University \textbf{THESIS}) 是根据《上海科技大学研究生学位论文撰写规范(初稿)》和《上海科技大学本科毕业论文(设计)工作条例(试行)》(下文统一简称《规范》)编写的、适用于上海科技大学学位论文写作的\emph{非官方} \LaTeX 模板。目前版本(\fileversion{})提供了本科、硕士和博士学位论文排版选项,且能够自动生成用于盲审的匿名版以及最终提交的打印版论文。项目主页:\url{https://github.com/lirundong/shtthesis}
% \end{abstract}
%
% \vskip\parskip
%
% \def\abstractname{排版样式说明}
% \begin{abstract}
% 本文档针对各部分不同内容使用不同的排版样式:文档正文使用宋体和英文衬线体(serif),\emph{强调部分}使用\emph{楷体}和英文意大利体(\emph{italic}),宏包名称使用英文无衬线体(\textsf{sans serif},例如 \textsf{hyperref}),代码及选项使用英文等宽体(\texttt{typewriter})和\texttt{等宽细黑体}排版。
%
% 文中以 |$| 开头的代码块表示终端命令,例如
% \begin{shell}
% `\prompt' latexmk shtthesis.tex
% \end{shell}
% 使用时不必输入 |$| 字符。其他代码块表示 \LaTeX{} 命令,例如
% \begin{latex}
% \foo`\oarg{argi}\marg{argii}'
% \end{latex}
% 其中,由 |[]| 包裹的为命令的\emph{可选参数},由 |{}| 包裹的为命令的\emph{必选参数},由 |<>| 包裹的为\emph{参数名称},调用时不必输入参数前后的尖括号|<>|。例如
% \begin{latex}
% \foo[bar]{baz}
% \end{latex}
% \end{abstract}
%
% \vskip\parskip
%
% \def\abstractname{注意}
% \begin{abstract}
% \noindent
% \begin{enumerate}
% \item 本文档排版效果与 \shtthesis{} 文档类排版效果无关;
% \item 本文档并非 \LaTeX{} 入门教程。若您尚不熟悉 \LaTeX{},推荐您阅读 \textsf{lshort}~\cite{oetiker2018not} 和它的中文翻译版 \textsf{lshort-zh-cn}~\cite{ctex2020not},或是\citeauthor{liu2013latex}编著的《\LaTeX{} 入门》~\cite{liu2013latex};
% \item \shtthesis{} 尚未进入稳定版且仍在快速迭代中,每次副版本号升级($\symup{v}0.x.y \to \symup{v}0.x+1.z$)不保证前向兼容。若您已开始写作,升级版本需非常谨慎;
% \item \shtthesis{} 项目使用 \href{https://www.gnu.org/licenses/gpl-3.0.html}{GNU General Public License v3} 分发;
% \end{enumerate}
% \end{abstract}
%
% \clearpage
% \begin{multicols}{2}[
%   \setlength{\columnseprule}{.4pt}
%   \setlength{\columnsep}{18pt}]
%   \tableofcontents
% \end{multicols}
%
% \clearpage
% \section{模板安装}
% \shtthesis{} 已发布至 \href{https://www.ctan.org/pkg/shtthesis}{CTAN} 且被收录于 \TeX{} Live,推荐使用 \TeX{} Live 包管理器 |tlmgr| 直接安装:
% \begin{shell}
% `\prompt' tlmgr install shtthesis
% \end{shell}
% 若当前发行版已包含 \shtthesis{},建议在使用前更新:
% \begin{shell}
% `\prompt' tlmgr update shtthesis
% \end{shell}
% 
% 为避免版权问题,上传至 CTAN 的 \shtthesis{} 并不包含校徽文件,需要至项目主页下载 \href{https://github.com/lirundong/shtthesis/raw/master/shanghaitech-emblem.pdf}{shanghaitech-emblem.pdf}。若用户论文文档为 thesis.tex,参考文献数据库为 reference.bib,则需将下载的校徽文件与它们放在同一目录下,下文称为\emph{工作目录}。工作目录中必要的文件包括:
% \begin{center}
%   \begin{tabular}{ll}
%     \toprule
%     文件名称 & 说明 \\
%     \midrule 
%     thesis.tex & 论文文档 \\
%     reference.bib & 参考文献数据库 \\
%     shanghaitech-emblem.pdf & 上海科技大学校徽 \\
%     \bottomrule
%   \end{tabular}
% \end{center}
% 
% \subsection{文档编译}
% \shtthesis{} 支持使用 \hologo{XeTeX} 和 \hologo{LuaTeX} 引擎编译(注意,\emph{不支持} \hologo{pdfTeX})。推荐在最新的 \hologo{TeX} Live 环境下,使用 |latexmk| 工具进行编译。Windows 及 Linux 用户请下载安装 \href{https://www.tug.org/texlive/}{\hologo{TeX} Live},macOS 用户请下载安装 \href{https://www.tug.org/mactex/}{Mac\hologo{TeX}}。\emph{非常不推荐}使用 $\mathbb{C}$\TeX 发行版(\emph{大人,时代变了})。
% 
% 在完成环境配置后,即可使用 |latexmk| 工具完成编译。打开终端(Windows 用户打开 CMD)切换至工作目录,使用 \hologo{XeTeX} 引擎进行编译:
% \begin{shell}
% `\prompt' latexmk -pdfxe
% \end{shell}
% 若偏好使用 \hologo{LuaTeX} 引擎,则编译命令为:
% \begin{shell}
% `\prompt' latexmk -pdflua
% \end{shell}
% 
% 一般来说,\hologo{XeTeX} 引擎的编译速度较快且占用资源较少,而 \hologo{LuaTeX} 引擎的编译结果似乎有更好的跨平台规范性。
%
% \section{模板设定}
% \subsection{载入\shtthesis 模板类}
% 模板安装完成后,在论文文件 thesis.tex 开头使用
% \begin{latex}
% \documentclass`\oarg{类参数}'{shtthesis}
% \end{latex}
% 即可载入模板。\meta{类参数}包括:用于指定学位类型的 |bachelor|、 |master|、 |doctor| 选项,用于生成盲审论文的 |anonymous| 选项,用于生成打印版论文的 |print| 选项,以及传递给 \textsf{cexbook} 的其他选项。
% 
% \subsubsection{指定学位类型}
% \DescribeMacro{bechelor}
% \DescribeMacro{master}
% \DescribeMacro{doctor}
% \shtthesis{} \fileversion{} 接受的学位类型包括学士(bachelor)、硕士(master)和博士(doctor),学位类型必须指定且不能同时指定多种学位。指定学位类型后,\shtthesis{} 会按照相应排版规则,生成不同的封面和正文格式。
% 
% 需要注意的是,教务处给出的本科生论文排版规范~\citep{bachelor2019} 中包含一些非常费解和不一致的内容。考虑到 2020 年学位申请时,本科生收到过下述通知:
% \begin{quotation}
% “如果是使用Latex版模板,保证封面与学校的word版要求完全一致,正文中的格式不要求在字号等细节方面与Word版要求严格一致,但要做到自己的论文自统一并且符合常用规范。”
% \end{quotation}
% \DescribeMacro{comfort}
% 因此 \shtthesis{} 为本科生论文额外提供了 |comfort| 选项,在保证封面符合规范的同时使论文排版更为舒适且正式。推荐排版本科论文时使用:
% \begin{latex}
% \documentclass[bachelor, comfort]{shtthesis}
% \end{latex}
% 
% \subsubsection{\texttt{anonymous} 选项} \label{sec::option_anonymous}
% \DescribeMacro{anonymous}
% 为方便生成研究生论文盲审时所需的匿名版论文,传入 |anonymous| 类参数即可将论文中英文封面的作者信息、导师信息,以及附录中的作者简历替换为匿名字符串,将作者论文发表、专利申请记录替换为匿名版本,并隐去文末的致谢部分:
% \begin{latex}
% \documentclass[anonymous]{shtthesis}
% \end{latex}
%
% \DescribeMacro{anonymous-str}
% \shtthesis{} 默认的匿名字符串为连续的三个英文星号“***”,该字符串可使通过 |\shtsetup| 的 |anonymous-str| 选项修改:
% \begin{latex}
% \shtsetup{
%   anonymous-str = XXX, % 作者、导师姓名在匿名环境下显示为 XXX
% }
% \end{latex}
% |\shtsetup| 命令的使用方法详见第~\ref{sec::setup_info} 节。
% 
% \subsubsection{\texttt{print} 选项}
% \DescribeMacro{print}
% 传入 |print| 选项后,论文中所有超链接变为黑色(包括文献引用、URL 等),以避免黑白打印后彩色链接内容呈浅灰色影响观感;同时将奇数页、偶数页的内侧页边距分别增加 0.63 厘米以便于装订。
% \begin{latex}
% \documentclass[print]{shtthesis}
% \end{latex}
% 
% \subsubsection{传递给 \textsf{ctexbook} 的其他选项}
% \shtthesis{} 实际使用 \CTeX 宏包提供的 \textsf{ctexbook} 文档类排版,除上述选项外的类参数会传递给 \textsf{ctexbook}。其中需要注意的选项为 |fontset|,即设定论文所用的字体集。\CTeX 宏包自身能够根据编译平台选择合适的字体集,也可以手动设置相应的 fontset,例如在 Windows 平台下设置 |fontset=windows|,在 macOS 平台下设置 |fontset=mac|:
% \DescribeMacro{fontset}
% \begin{latex}
% \documentclass[fontset=windows]{shtthesis}
% \end{latex}
% 不同字体集所用字体见表~\ref{tab::fonts}。在 Linux/UNIX 环境下对于宋体和黑体,会首先检测思源宋体/黑体(Source Han 字体或 Noto CJK 字体)是否安装,若未检测到则回退至 Fandol 宋体/黑体;对于楷体和仿宋,则会依次检测方正楷体/仿宋-GBK 是否安装,若未检测到则退回至 Fandol 楷体/仿宋。
% 
% \begin{table}[htb]
%   \centering
%   \caption{不同字符集下 \shtthesis{} 所用字体}
%   \label{tab::fonts}
%   \begin{subtable}{\columnwidth}
%     \centering
%     \caption{\shtthesis{} 所用中文字体}
%     \label{tab::chs_fonts}
%     \begin{tabular}{*{3}{c}}
%       \toprule
%       Windows & macOS & Linux/UNIX \\
%       \midrule
%       \songti   中易宋体 & \songti   华文宋体简体 & \songti   思源宋体 $\to$ Fandol 宋体 \\
%       \heiti    中易黑体 & \heiti    华文黑体简体 & \heiti    思源黑体 $\to$ Fandol 黑体 \\
%       \kaishu   中易楷体 & \kaishu   华文楷体简体 & \kaishu   方正楷体-GBK $\to$ Fandol 楷体 \\
%       \fangsong 中易仿宋 & \fangsong 华文仿宋简体 & \fangsong 方正仿宋-GBK $\to$ Fandol 仿宋 \\
%       \bottomrule
%     \end{tabular}
%   \end{subtable}
%   \newline
%   \vspace{12pt}
%   \newline
%   \begin{subtable}{\columnwidth}
%     \centering
%     \caption{\shtthesis{} 所用英文字体(全平台一致)}
%     \label{tab::eng_fonts}
%     \begin{tabular}{*{3}{c}}
%       \toprule
%       \textrm{Serif} & \textsf{Sans Serif} & \texttt{Typewriter} \\
%       \midrule
%       \textrm{\TeX{} Gyre Termes} & \textsf{\TeX{} Gyre Heros} & \texttt{\TeX{} Gyre Cursor} \\
%       \bottomrule
%     \end{tabular}
%   \end{subtable}
% \end{table}
% 
% 需要注意 Fandol 系列字体虽然为 \TeX{} Live 自带,但其字符覆盖有限,对于生僻字可能出现缺字情况。思源宋体\footnote{\url{https://source.typekit.com/source-han-serif/cn/}}、思源黑体\footnote{\url{https://github.com/adobe-fonts/source-han-sans}}、方正楷体-GBK\footnote{\url{https://www.foundertype.com/index.php/FontInfo/index/id/137}}和方正仿宋-GBK\footnote{\url{https://www.foundertype.com/index.php/FontInfo/index/id/128}} 均为免费字体且完整覆盖简繁扩展 (GBK) 字符集,非常推荐 Linux/UNIX 用户安装使用。
% 
% \subsection{设定论文必要信息} \label{sec::setup_info}
% \DescribeMacro{\shtsetup}
% 知晓学位类型后,\shtthesis{} 还需学位名称、专业、作者及导师信息等其他信息才能进行进一步排版。以上均可通过 |\shtsetup| 命令,在论文导言区(即 |\documentclass| 之后、|\begin{document}| 之前)以 key=value 方式统一设定。用户可以一次性调用 |\shtsetup| 设定所有信息,也可分多次设定。表示中文信息条目的 key 为 |-| 连接的小写单词(例如研究生学位中文名称 |degree-name|),由 |*| 结尾的 key 一般表示对应的英文条目(例如研究生学位英文名称 |degree-name*|);value 可以在前后添加大括号,也可以不添加。特别需要注意 |\shtsetup| 命令内\emph{不能有空行}。
%
% \subsubsection{学位信息}
% \DescribeMacro{degree-name}
% \DescribeMacro{degree-name*}
% 研究生学位论文需要额外设定的学位信息包括:学位名称(degree-name)和英文学位名称(degree-name*),具体见表~\ref{tab::degree_info}。学位类型会影响 \shtthesis{} 的中英文封面内容。
% \begin{table}[htb]
% \centering
% \caption{需额外设定的\emph{研究生}学位信息} \label{tab::degree_info}
% \begin{tabular}{ll}
%   \toprule
%   degree-name & degree-name* \\
%   \midrule
%   学术型博士 & Doctor~of~Philosophy \\
%   理学硕士 & Master~of~Natural~Science \\
%   工学硕士 & Master~of~Science~in~Engineering \\
%   \bottomrule
% \end{tabular}
% \end{table}
% 
% \subsubsection{论文标题信息}
% \DescribeMacro{title}
% \DescribeMacro{title*}
% \DescribeMacro{secret-level}
% 设定论文中英文标题和涉密等级。中文标题(title)和英文标题(title*)中可以包含换行符。研究生学位论文的涉密等级(secret-level)会以“密级:\uline{XXX}”显示在中文封面右上角,未设定涉密等级则不显示相关信息。
% 
% \subsubsection{作者信息}
% \DescribeMacro{discipline}
% \DescribeMacro{discipline*}
% \DescribeMacro{discipline-level-1}
% \DescribeMacro{discipline-level-1*}
% 需要设定的作者信息包括:作者中英文姓名(author 和 author*)和学校/学院中英文名称(institution 和 institution*),研究生需要设定一级学科中英文名称(discipline-level-1 和 discipline-level-1*),本科生需要设定攻读专业中英文名称(discipline 和 discipline*)。
% 
% 特别注意研究生和本科生的英文名格式要求不一致:研究生要求以姓氏拼音在前、名字拼音在后、中间以半角空格分隔:
% \DescribeMacro{author}
% \DescribeMacro{author*}
% \begin{latex}
% \shtsetup{
%   % 研究生中英文姓名格式要求
%   author = 李润东,
%   author* = {Li~Rundong}, % 正确写法
%   % author* = {Rundong~Li}, % 错误写法
% }
% \end{latex}
% 而本科生要求以名字拼音在前、姓氏拼音在后、中间以半角空格分隔:
% \begin{latex}
% \shtsetup{
%   % 本科生中英文姓名格式要求
%   author = 李润东,
%   author* = {Rundong~Li}, % 正确写法
%   % author* = {Li~Rundong}, % 错误写法
% }
% \end{latex}
% 
% 研究生需要设定完整的学校、学院名称,中文名称(institution)不可带换行符,英文名称(institution*)应在学院、学校名称间加入换行符,否则封面排版会出现错误:
% \DescribeMacro{institution}
% \DescribeMacro{institution*}
% \begin{latex}
% \shtsetup{
%   institution = 上海科技大学信息科学与技术学院,
%   institution* = {School~of~Information~Science~and~Technology\\%
%                   ShanghaiTech~University},
% }
% \end{latex}
% 本科生则不必包含学校名称,直接设定学院名称即可,注意中英文名称中均不能包含换行符:
% \begin{latex}
% \shtsetup{
%   institution = 信息科学与技术学院,
%   institution* = {School~of~Information~Science~and~Technology},
% }
% \end{latex}
% 
% \subsubsection{导师信息}
% \DescribeMacro{supervisor}
% \DescribeMacro{supervisor*}
% \DescribeMacro{supervisor-institution}
% 需要设定的导师信息包括:导师中英文姓名(supervisor 和 supervisor*)和导师单位中文名称(supervisor-institution,仅需研究生论文设定)。在研究生论文中文封面中,导师姓名和单位分两行显示在“指导教师”条目下。导师中文姓名应包含导师职称,如教授、副教授、助理教授、研究员、副研究员,以一个半角空格跟随在姓名之后;导师英文姓名则统一以“Professor”开头,与英文姓名以一个半角空格隔开。导师单位中文名称不可包含换行符。
% \begin{latex}
% \shtsetup{
%   supervisor = {范睿~副教授},
%   supervisor* = {Professor~Fan~Rui},
%   supervisor-institution = {上海科技大学信息科学与技术学院},
% }
% \end{latex}
% 
% \subsubsection{成文日期}
% \DescribeMacro{date}
% \DescribeMacro{date*}
% 通过 date 和 date* 分别设置中英文成文日期,日期内容应按照申请学位的时间节点填写,具体请参考当年《规范》。研究生成文日期格式为:
% \begin{latex}
% \shtsetup{
%   date = {2020~年~6~月},
%   date* = {June,~2020},
% }
% \end{latex}
% 本科生成文日期格式为:
% \begin{latex}
% \shtsetup{
%   date = {2020~年~06~月},
%   date* = {06~/~2020},
% }
% \end{latex}
% 
% \section{论文内容}
% \subsection{前言部分}
% 论文前言部分应包含论文原创性声明和授权使用声明、中英文摘要、目录,研究生论文还需包含图形、表格列表。若有需要,可加入符号列表。\shtthesis{} 会自动生成目录和图表列表,在不设置 |anonymous| 选项时会自动生成声明页。\shtthesis{} 提供了摘要和符号列表环境,以便于用户生成中英文摘要和符号列表。
% 
% \subsubsection{论文摘要及关键词}
% \DescribeEnv{abstract}
% \DescribeEnv{abstract*} 
% 论文中英文摘要在正文部分 |\frontmatter| 之后、|\makeindices| 之前,分别在 abstract 和 abstract* 环境中输入。摘要环境对内容没有限制:
% \begin{latex}
% \begin{abstract}
%   中文摘要内容...
% \end{abstract}
% 
% \begin{abstract*}
%   English abstract content...
% \end{abstract*}
% \end{latex}
% 
% \DescribeMacro{flattitle}
% 需要注意,本科生论文要求在中英文摘要中包含论文标题。如果希望在摘要中显示的标题不换行,可以指定 |flattitle| 选项,排版时会将 title 和 title* 中的换行符替换为空格:
% \begin{latex}
% \begin{abstract}[flattitle]
%   % 摘要标题排版时不换行
%   中文摘要内容...
% \end{abstract}
% \end{latex}
% 
% \DescribeMacro{keywords}
% \DescribeMacro{keywords*}
% 论文中英文关键词在 |\shtsetup| 命令中分别以 keywords 和 keywords* 设定。注意 \shtthesis{} \fileversion{} 中尚未实现分词处理,此处输入的 value 将不经任何预处理,直接插入至正文中排版。因此,中文关键词(keywords)之间应该以中文逗号“,”隔开且不包含空格;英文关键词(keywords*)之间应该以半角逗号“,”隔开,并在每一半角逗号后跟随一个半角空格。
% \begin{latex}
% \shtsetup{
%   keywords = {上海科技大学,学位论文,\LaTeX{}},
%   keywords* = {ShanghaiTech~University, Thesis, \LaTeX{}},
% }
% \end{latex}
% 
% \subsubsection{目录及图形、表格、符号列表}
% \DescribeMacro{\makeindices}
% \shtthesis{} 重载了 |\makeindices| 命令,可一次生成目录、图形列表和表格列表。默认目录中正文标题列出至第 3 级(即 subsection),若包含附录则目录中只列出“附录”一项(附录各级标题仍会生成 PDF 书签),图形、表格列表中列出图表标题的全部内容。对于特别长的图形、表格标题,可以在 |\caption| 命令中指定短标题,从而使列表条目更为简明:
% \begin{latex}
% \begin{figure}
%   \caption`\oarg{出现在图形列表内的短标题}\marg{出现在正文中的长标题}'
%   % ...
% \end{figure}
% \end{latex}
% 
% \DescribeEnv{nomenclatures}
% \shtthesis{} 提供了 nomenclatures 环境用于在研究生论文中生成符号列表,其使用方法为:
% \begin{latex}
% \begin{nomenclatures}`\oarg{标题}'
%   \head`\oarg{单位表头}\marg{符号表头}\marg{描述表头}'
%   \item`\oarg{单位}\marg{符号}\marg{描述}'
%   % ...
% \end{nomenclatures}
% \end{latex}
% 每次 nomenclatures 调用会分别生成一组符号列表,每一 |\item| 会生成一行符号描述。若 \meta{标题} 不为空则会为当前组生成一个不编号的 section 级的标题。|\head| 为可选指令,会为当前组生成一个“表头”,但必须在 |\item| 之前列出。|\item| 接受一个可选参数 \meta{单位} 作为符号的单位,若其不为空则在当前行右对齐出现。nomenclatures 的使用样例可以参考 \jobname.tex 中“符号列表”部分。注意在 \shtthesis{} \fileversion{} 中各 |\item| 的 \meta{描述} 在折行后会出现排版错误,故暂时不支持较长的符号描述。
% 
% \subsection{正文部分}
% \shtthesis{} 正文部分的书写与一般 \LaTeX 项目无异。\shtthesis{} 为研究生论文修改公式编号格式以符合《规范》要求,并提供编号定理、证明等常用数学环境。文献引用遵循 GB/T 7714-2015 《信息与文献 参考文献著录规则》标准。
% 
% \subsubsection{编号公式}
% 在排版研究生论文时,\shtthesis{} 通过 \textsf{mathtools} 宏包在公式编号前添加 $\ldots$ 以符合《规范》对公式编号的格式要求,如:
% \begin{equation}
% \usetagform{dots}
% 𝑃(𝐴∣𝐵) = \frac{𝑃(𝐴)𝑃(𝐵∣𝐴)}{𝑃(𝐵)} \label{eq::bayesian}
% \end{equation}
% 同时重载了 |\eqref|,使得公式编号格式修改后,其引用格式仍与 \textsf{amsmath} 无异:贝叶斯定理~\eqref{eq::bayesian}。排版本科生论文时不修改公式编号格式。
% 
% \shtthesis{} 使用 \textsf{unicode-math} 宏包进行公式排版,因此在数学环境内既可以用标准 \LaTeX{} 宏,也可以直接输入 Unicode 符号。例如 $\oiint$ 符号可以通过 |$\oiint$| 宏录入,也可以通过 Unicode 符号 |$|$∯$|$| (对应 |U+0222F| 码点) 录入。\textsf{unicode-math} 支持的符号可以通过
% \begin{shell}
% `\prompt' texdoc unimath-symbols
% \end{shell}
% 打开 \textsf{unicode-math} symbols 文档查找和复制录入。
% 
% \subsubsection{数学环境}
% \DescribeEnv{theorem}
% \DescribeEnv{lemma}
% \DescribeEnv{corollary}
% \DescribeEnv{proposition}
% \DescribeEnv{conjecture}
% \DescribeEnv{definition}
% \DescribeEnv{axiom}
% \DescribeEnv{example}
% \DescribeEnv{problem}
% \DescribeEnv{exercise}
% \DescribeEnv{remark}
% \DescribeEnv{proof}
% \shtthesis 通过 \textsf{ntheorem} 宏包定义了常用的数学环境和证明环境,如表~\ref{tab::math_envs} 所列。其中,英文表示 tex 文档内调用的环境名称,中文表示排版后论文中显示的环境名称。注意,在使用 \hologo{LuaTeX} 引擎编译时,证明环境结尾的 QED 符号 $\QED$ 可能会排版出错。
% 
% \begin{table}[htb]
% \centering
% \caption{\shtthesis 提供的数学环境}
% \label{tab::math_envs}
% \begin{tabular}{*{5}{c}}
%   \toprule
%   theorem & lemma & corollary & proposition & conjecture \\
%   定理    & 引理  & 推论       & 命题        & 猜想       \\
%   \midrule
%   definition & axiom & example & problem & exercise \\
%   定义       & 公理  & 例       & 问题    & 练习     \\
%   \midrule
%   remark & proof & & & \\
%   注 & 证明 & & & \\
%   \bottomrule
% \end{tabular}
% \end{table}
% 
% \subsubsection{文献引用} \label{sec::citation}
% \DescribeMacro{\makebiblio}
% \shtthesis{} 通过 \textsf{biblatex} 宏包支持符合 GB/T 7714-2015 标准的文献引用。本科生论文使用顺序编码制,研究生论文使用“著者-出版年”制。参考文献数据库通过 |\shtsetup| 的 bib-resource 选项指定,正文参考文献章节通过 |\makebiblio| 命令生成:
% \begin{latex}
% % preamble
% \shtsetup{
%   bib-resource = {reference.bib},
% }
% % document
% \makebiblio
% \end{latex}
% 
% \paragraph{顺序编码制}
% \DescribeMacro{\cite}
% \DescribeMacro{\inlinecite}
% 引用条目排版为数字编号,参考文献列表按照各条目在正文中被引用顺序排列。
% \begin{center}
% \begin{tabular}{ll}
% \toprule
% |\cite{liu2013latex}|& 上标引用~\cite{liu2013latex}\\
% |\inlinecite{liu2013latex}|& 行内引用~\parencite{liu2013latex}\\
% \bottomrule
% \end{tabular}
% \end{center}
% 
% \paragraph{“著者-出版年”制}
% \DescribeMacro{\citet}
% \DescribeMacro{\citep}
% \DescribeMacro{\citeauthor}
% \DescribeMacro{\citeyear}
% 引用方式与 \textsf{natbib} 一致,参考文献列表各条目按照语言分组后,按照字母序排列。
% \begin{center}
% \begin{tabular}{ll}
% \toprule
% |\citet{liu2013latex}|& 文本引用 \citeauthor{liu2013latex} (\citeyear{liu2013latex})\\
% |\citep{liu2013latex}|& 括号引用 (\citeauthor{liu2013latex}, \citeyear{liu2013latex})\\
% |\citeauthor{liu2013latex}|& 引用作者 \citeauthor{liu2013latex}\\
% |\citeyear{liu2013latex}|& 引用年份 \citeyear{liu2013latex}\\
% \bottomrule
% \end{tabular}
% \end{center}
% 
% 需要注意若文献条目的作者姓名非英文,则需额外增加 key 字段指定作者英文姓名,并在姓氏拼英后加注音调,以确保文献条目在文末的参考文献中正确排列。例如:
% \begin{latex}
% @incollection{chen1980zhongguo,
%   author = {陈晋镳 and 张惠民 and 朱士兴 and 赵震 and 王振刚},
%   key    = {Chen2 Jing Ao Zhang1 Hui Ming Zhu1 Shi Xing Zhao4 Zhen Wang2 Zhen Gang},
%   % ...
% }
% \end{latex}
% 
% \subsubsection{列表环境}
% \DescribeEnv{enumerate}
% \DescribeEnv{itemize}
% \DescribeEnv{description}
% \shtthesis{} 微调了编号列表(enumerate)、非编号列表(itemize)和关键字列表(description)环境以适应中文排版惯例。编号列表默认使用数字编号,可以使用短标签形式指定编号格式(例如 |[a)]| 切换至小写字母+半角括号编号)。还可以通过 |resume*| 参数“继续”被中断的列表编号。关键字列表环境在每一条目关键字后增加了加粗全角冒号“\textbf{:}”以适应中文排版惯例。
% 
% \subsection{附录及后记} \label{sec::backmatter}
% \DescribeMacro{\appendix}
% 正文完成后,使用 |\appendix| 命令后即可切换至附录模式。默认目录中只显示“附录”一项,不显示附录各级标题。
% \begin{latex}
% \appendix
% \chapter{...}
% % ...
% \end{latex}
% 
% 《规范》要求在文末依此列出致谢、作者简历、攻读学位期间发表的论文与研究成果。\shtthesis{} 提供了 acknowledgement、resume、publications、patterns 和 projects 排版环境。同时为了生成符合盲审要求的论文,\shtthesis{} 也提供了对应的\emph{匿名环境} publications*、patterns* 和 projects*。在打开 |anonymous| 选项(第~\ref{sec::option_anonymous} 节)后,论文中不出现“致谢”一节,作者简历内容替换为匿名字符串,其他节使用匿名环境内容排版。注意在第一次使用任一上述环境前,需要使用 |\backmatter| \DescribeMacro{\backmatter} 切换至后记模式。
% 
% \subsubsection{致谢}
% 在 acknowledgement 环境内书写致谢,致谢内容在匿名模式下不显示。
% \DescribeEnv{acknowledgement}
% \begin{latex}
% \begin{acknowledgement}
%   致谢信息……
% \end{acknowledgement}
% \end{latex}
% 
% \subsubsection{简历及科研成果}
% \DescribeEnv{resume}
% \DescribeEnv{publications}
% \DescribeEnv{publications*}
% \DescribeEnv{patents}
% \DescribeEnv{patents*}
% \DescribeEnv{projects}
% \DescribeEnv{projects*}
% 此部分需要依此书写个人简历(resume 环境)、已发表(或正式接受)的学术论文(publications 和 publications* 环境)、申请或已获得的专利(patents 和 patents* 环境)、参加的研究项目及获奖情况(projects 和 projects*)。根据是否匿名分别显示非匿名环境内容和匿名环境内容。
% \begin{latex}
% \begin{resume}
%   个人简历…… (仅非匿名环境显示)
% \end{resume}
%
% \begin{publications}
%   论文发表记录…… (非匿名时显示)
% \end{publications}
% 
% \begin{publications*}
%   论文发表记录…… (匿名时显示)
% \end{publications*}
% \end{latex}
%
% \printbibliography
%
% \iffalse sample bibliography database
%<*bib>
@online{bachelor2019,
  author = {上海科技大学教学事务处},
  key = {shang hai ke ji da xue jiao xue shi wu chu},
  title = {本科毕业论文(设计)表格模板下载(任务书、封面、撰写模板等)},
  year = 2019,
  url = {http://oaa.shanghaitech.edu.cn/2019/0321/c4666a41070/page.htm},
  urldate = {2020-06-17}
}
@online{clerkma2013unicode,
  author = {Clerk Ma},
  title = {如何在{XeTeX}中单独设置数学字体,为什么{STIX}的数学字体很牛?},
  year = 2013,
  url = {https://www.zhihu.com/question/20592491/answer/15577847},
  urldate = {2020-06-30}
}
@book{liu2013latex,
  title={{\LaTeX}入门},
  author={刘海洋},
  key={liu2 hai yang},
  publisher={电子工业出版社},
  address={北京},
  year={2013},
}
@online{oetiker2018not,
  author={Oetiker, Tobias and Partl, Hubert and Hyna, Irene and Schlegl, Elisabeth},
  title={The not so short introduction to {\LaTeX}2$\varepsilon$},
  year={2018},
  url = {http://mirrors.ctan.org/info/lshort/english/lshort.pdf},
  urldate = {2020-07-16},
}
@online{ctex2020not,
  author={{C\TeX{}}开发小组},
  key = {ctex kai fa xiao zu},
  title={一份(不太)简短的 {\LaTeX}2$\varepsilon$ 介绍},
  year={2020},
  url = {http://mirrors.ctan.org/info/lshort/chinese/lshort-zh-cn.pdf},
  urldate = {2020-07-16},
}
%</bib>
% \fi
% 
% \StopEventually{\PrintChanges\PrintIndex}
% \appendix
% \clearpage
% \section{主模板类 \shtthesis{} 实现}
%
% \subsection{前期准备}
% \subsubsection{文档类信息和工具宏}
% \begin{macro}{\versiondate}
% \begin{macro}{\version}
% 设定 \shtthesis{} 宏包版本号与版本日期。
%    \begin{macrocode}
%<cls|doc>\NeedsTeXFormat{LaTeX2e}
%<*cls>
\newcommand\versiondate{2020/07/17}
\newcommand\version{0.3.2-dev}
\ProvidesClass{shtthesis}[%
  \versiondate\space%
  v\version\space%
  An Unofficial Thesis Template for ShanghaiTech University]
%</cls>
%    \end{macrocode}
% \end{macro}
% \end{macro}
%
% \begin{macro}{\ShtThesis}
% \begin{macro}{\shtthesis}
%  提供简单的 |\shtthesis| 命令以排版宏包名称。
%    \begin{macrocode}
%<*(cls|doc)>
\newcommand\ShtThesis{\textup{\sffamily ShtThesis}}
\newcommand\shtthesis{\textup{\sffamily shtthesis}}
\hyphenation{sht-thesis}
\hyphenation{Sht-Thesis}
%</(cls|doc)>
%    \end{macrocode}
% \end{macro}
% \end{macro}
%
% 准备文档类报错信息和内部工具宏,并检查编译引擎是否为 \hologo{XeTeX} 或 \hologo{LuaTeX}。
%    \begin{macrocode}
%<*cls>
\def\sht@compile@err@help{%
  Pass `-pdflua' or `-pdfxe' option to `latexmk' on compilation.%
}
\newcommand\sht@error[2][\sht@compile@err@help]{\ClassError{shtthesis}{#2}{#1}}
\newcommand\sht@warning[1]{\ClassWarning{shtthesis}{#1}}
\RequirePackage{iftex}
\ifluatex\else\ifxetex\else%
  \sht@error{shtthesis only works with LuaTeX or XeTeX}%
\fi\fi
%</cls>
%    \end{macrocode}
%
% \subsubsection{\texttt{shtsetup} 用户接口}
% \begin{macro}{\shtsetup}
% \fileversion{} 的 |\shtsetup| 代码实现修改自 \href{https://github.com/tuna/thuthesis/blob/17acabec987947768e35f0693c678ce654020bc2/thuthesis.dtx#L1133-L1349}{thuthesis},非常感谢薛瑞尼教授和 TUNA 的工作!
%    \begin{macrocode}
%<*cls>
\RequirePackage{kvdefinekeys}
\RequirePackage{kvsetkeys}
\RequirePackage{datetime}
%</cls>
%    \end{macrocode}
%
% 主要用户接口:
%    \begin{macrocode}
%<*cls>
\newcommand\shtsetup{%
  \kvsetkeys{sht}%
}
%</cls>
%    \end{macrocode}
%
% 定义 keys 的开发接口:
%    \begin{macrocode}
%<*cls>
\newcommand\sht@define@key[1]{%
  \kvsetkeys{sht@key}{#1}%
}
%</cls>
%    \end{macrocode}
%
% \textsf{kvsetkeys} 的 handler function,v0.4.x 会用 \LaTeX3 重构:
%    \begin{macrocode}
%<*cls>
\kv@set@family@handler{sht@key}{%
  \@namedef{sht@#1@@name}{#1}%
  \def\sht@@default{}%
  \def\sht@@choices{}%
  \kv@define@key{sht@value}{name}{%
    \@namedef{sht@#1@@name}{##1}%
  }%
  \kv@define@key{sht@value}{code}{%
    \@namedef{sht@#1@@code}{##1}%
  }%
  \@namedef{sht@#1@@check}{}%
  \@namedef{sht@#1@@code}{}%
  \@namedef{sht@#1@@hook}{%
    \expandafter\ifx\csname\@currname.\@currext-h@@k\endcsname\relax
      \@nameuse{sht@#1@@code}%
    \else
      \AtEndOfClass{%
        \@nameuse{sht@#1@@code}%
      }%
    \fi
  }%
  \kv@define@key{sht@value}{choices}{%
    \def\sht@@choices{##1}%
    \@namedef{sht@#1@@reset}{}%
    \@namedef{sht@#1@@check}{%
      \@ifundefined{%
        ifsht@\@nameuse{sht@#1@@name}@\@nameuse{sht@\@nameuse{sht@#1@@name}}%
      }{%
        \sht@error{Invalid value "#1 = \@nameuse{sht@\@nameuse{sht@#1@@name}}"}%
      }%
      \@nameuse{sht@#1@@reset}%
      \@nameuse{sht@\@nameuse{sht@#1@@name}@\@nameuse{sht@\@nameuse{sht@#1@@name}}true}%
    }%
  }%
  \kv@define@key{sht@value}{default}{%
    \def\sht@@default{##1}%
  }%
  \kvsetkeys{sht@value}{#2}%
  \@namedef{sht@\@nameuse{sht@#1@@name}}{}%
  \kv@set@family@handler{sht@choice}{%
    \ifx\sht@@default\@empty
      \def\sht@@default{##1}%
    \fi
    \expandafter\newif\csname ifsht@\@nameuse{sht@#1@@name}@##1\endcsname
    \expandafter\g@addto@macro\csname sht@#1@@reset\endcsname{%
      \@nameuse{sht@\@nameuse{sht@#1@@name}@##1false}%
    }%
  }%
  \kvsetkeys@expandafter{sht@choice}{\sht@@choices}%
  \expandafter\let\csname sht@\@nameuse{sht@#1@@name}\endcsname\sht@@default
  \expandafter\ifx\csname sht@\@nameuse{sht@#1@@name}\endcsname\@empty\else
    \@nameuse{sht@#1@@check}%
    \@nameuse{sht@#1@@hook}%
  \fi
  \kv@define@key{sht}{#1}{%
    \@namedef{sht@\@nameuse{sht@#1@@name}}{##1}%
    \@nameuse{sht@#1@@check}%
    \@nameuse{sht@#1@@hook}%
  }%
}
%</cls>
%    \end{macrocode}
% 定义 |\shtsetup| keys 及默认 values:
%    \begin{macrocode}
%<*cls>
\sht@define@key{
  degree-name = {
    default = {工学博士},
    name    = degree@name,
  },
  degree-name* = {
    default = {Doctor~of~Philosophy},
    name    = degree@name@en,
  },
  language = {
    choices = {
      chinese,
      english,
    },
  },
  secret-level = {
    name = secret@level,
  },
  secret-year = {
    name = secret@year,
  },
  title = {
    default = {标题},
  },
  title* = {
    default = {Title},
    name    = title@en,
  },
  keywords,
  keywords* = {
    name = keywords@en,
  },
  author = {
    default = {姓名},
  },
  author* = {
    default = {Name of author},
    name    = author@en,
  },
  author-id = {
    default = {Id of author},
    name    = author@id,
  },
  entrance-year = {
    default = {Year of entrance},
    name    = entrance@year,
  },
  supervisor = {
    default = {导师姓名},
  },
  supervisor* = {
    default = {Name of supervisor},
    name    = supervisor@en,
  },
  supervisor-institution = {
    name = supervisor@institution,
  },
  supervisor-institution* = {
    name = supervisor@institution@en,
  },
  institution,
  institution* = {
    name = institution@en
  },
  discipline,
  discipline* = {
    name = discipline@en,
  },
  discipline-level-1 = {
    default = {一级学科名称},
    name    = discipline@level@i,
  },
  discipline-level-1* = {
    default = {Name of Level-one Discipline},
    name    = discipline@level@i@en,
  },
  discipline-level-2 = {
    default = {二级学科名称},
    name    = discipline@level@ii,
  },
  discipline-level-2* = {
    default = {Name of Level-two Discipline},
    name    = discipline@level@ii@en,
  },
  date = {
    default = {\the\year~年~\the\month~月},
  },
  date* = {
    default = {\monthname,~\the\year},
    name = date@en,
  },
  clc,
  udc,
  id,
  anonymous-str = {
    name = anonymous@str,
    default = {***},
  },
  bib-resource = {
    name = bib@resource,
  }
}
%</cls>
%    \end{macrocode}
% \end{macro}
%
% \subsubsection{解析文档类选项}
% \shtthesis{} 的文档类解析依赖于 \textsf{kvoptions}:
%    \begin{macrocode}
%<*cls>
\RequirePackage{kvoptions}
\SetupKeyvalOptions{
  family = sht,
  prefix = sht@,
}
%</cls>
%    \end{macrocode}
% 设定学位类型,学位必须被设定且只能设定一次。
%    \begin{macrocode}
%<*cls>
\newif\ifsht@undergraduate
\newif\ifsht@graduate
\newif\ifsht@degree@set
\sht@degree@setfalse
\newcommand\sht@check@degree@set{%
  \ifsht@degree@set%
    \sht@error{you can only set degree once}%
  \else%
    \sht@degree@settrue%
  \fi%
}
\DeclareVoidOption{bachelor}{%
  \sht@check@degree@set%
  \sht@undergraduatetrue%
  \sht@graduatefalse%
  \def\sht@degree{bachelor}%
}
\DeclareVoidOption{master}{%
  \sht@check@degree@set%
  \sht@undergraduatefalse%
  \sht@graduatetrue%
  \def\sht@degree{master}%
}
\DeclareVoidOption{doctor}{%
  \sht@check@degree@set%
  \sht@undergraduatefalse%
  \sht@graduatetrue%
  \def\sht@degree{doctor}%
}
%</cls>
%    \end{macrocode}
% 设定匿名选项、打印版选项和本科生论文的 comfort 选项,其他类选项传递给 \textsf{ctexbook}。最后检查学位设定。
%    \begin{macrocode}
%<*cls>
\DeclareBoolOption{anonymous}
\DeclareBoolOption{print}
\DeclareBoolOption{comfort}
\DeclareDefaultOption{\PassOptionsToClass{\CurrentOption}{ctexbook}}
\ProcessKeyvalOptions*
\ifsht@degree@set%
\else%
  \sht@error{you have not set degree (bachelor, master or doctor) yet}%
\fi%
%</cls>
%    \end{macrocode}
%
% \subsubsection{载入依赖宏包}
% \begin{quotation}
% 要来力!要来力!(指宏包风暴)
% \end{quotation}
% 本科生和研究生论文格式差别(字号、引用格式)需要作为类选项传递给 \textsf{ctexbook} 和 \textsf{biblatex}。
%    \begin{macrocode}
%<*cls>
\ifsht@undergraduate
  \PassOptionsToClass{zihao = 5}{ctexbook}
  \PassOptionsToPackage{style = gb7714-2015}{biblatex}
\else
  \PassOptionsToClass{zihao = -4}{ctexbook}
  \PassOptionsToPackage{style = gb7714-2015ay}{biblatex}
\fi
\PassOptionsToClass{
  openany, 
  scheme = plain,
}{ctexbook}
\PassOptionsToPackage{
  hyperref    = manual,
  gbpub       = false,
  gbcitelocal = chinese,
}{biblatex}
\LoadClass{ctexbook}
\RequirePackage{expl3}
\RequirePackage{xparse}
\RequirePackage[hyperref, table]{xcolor}
\RequirePackage{geometry}
\RequirePackage{calc}
\RequirePackage{verbatim}
\RequirePackage{etoolbox}
\RequirePackage{ifthen}
\RequirePackage{graphicx}
\RequirePackage{indentfirst}
\RequirePackage[normalem]{ulem}
\RequirePackage{fancyhdr}
\RequirePackage{pageslts}
\RequirePackage{tocvsec2}
\RequirePackage{letltxmacro}
\RequirePackage{fontspec}
\RequirePackage{caption}
\RequirePackage[shortlabels, inline]{enumitem}
\RequirePackage{mathtools}
\RequirePackage[amsmath, thmmarks]{ntheorem}
\RequirePackage[mathbf=sym, mathrm=sym]{unicode-math}
\RequirePackage{biblatex}
%</cls>
%    \end{macrocode}
% 确保 \textsf{hyperref} 最后被导入,之后手动打开 \textsf{biblatex} 的超链接,并使用等宽体排版超链接。链接配色方案来自 \href{https://github.com/stone-zeng/fduthesis/blob/b72fb0e/source/fduthesis.dtx#L1377}{fduthesis},非常优雅且现代。谢谢曾祥东的工作!
%    \begin{macrocode}
%<*cls>
\AtEndPreamble{
  \RequirePackage{hyperref}
  \addbibresource{\sht@bib@resource}
  \BiblatexManualHyperrefOn
  \hypersetup{
    pdfencoding       = auto,
    psdextra          = true,
    bookmarksnumbered = true,
    pdftitle          = {\sht@flat@title},
    pdfauthor         = {\sht@author},
  }
  \ifsht@print
    \hypersetup{
      hidelinks,
      colorlinks = false,
    }
  \else
    \definecolor{fdu@link}{HTML}{990000}
    \definecolor{fdu@url}{HTML}{0000B2}
    \definecolor{fdu@cite}{HTML}{007F00}
    \hypersetup{
      colorlinks = true,
      linkcolor  = fdu@link,
      urlcolor   = fdu@url,
      citecolor  = fdu@cite,
    }
  \fi
  \urlstyle{tt}
}
%</cls>
%    \end{macrocode}
%
% \subsubsection{设定其他宏和常量}
% 设定各类图表的中文标题,以及校徽文件缺失时显示在论文中的缺省报错信息:
%    \begin{macrocode}
%<*cls>
\def\@title{\sht@title}
\def\@author{\sht@author}
\def\contentsname{目\hspace{1\ccwd}录}
\def\listfigurename{图形列表}
\def\listtablename{表格列表}
\def\appendixname{附录}
\def\indexname{索引}
\def\refname{参考文献}
\def\bibname{参考文献}
\def\tablename{表}
\def\figurename{图}
\def\school@logo@missing{%
  校徽文件缺失,请至\href{https://github.com/lirundong/shtthesis/raw/master/%
  shanghaitech-emblem.pdf}{项目主页}下载!%
}
%</cls>
%    \end{macrocode}
% \begin{macro}{ShtRed}
% 颜色定义:上科大红
%    \begin{macrocode}
%<cls>\definecolor{ShtRed}{RGB}{146,46,23}
%    \end{macrocode}
% \end{macro}
%
% \begin{macro}{\sht@flat@title}
% \begin{macro}{\sht@flat@title@en}
% 定义不包含换行符的中英文标题。之后会在本科生论文摘要、页眉被使用。
%    \begin{macrocode}
%<*cls>
\def\sht@flat@title{\renewcommand\\{\space} \sht@title}
\def\sht@flat@title@en{\renewcommand\\{\space} \sht@title@en}
%</cls>
%    \end{macrocode}
% \end{macro}
% \end{macro}
%
% \subsection{页面、页眉及页脚设置}
% 使用 \textsf{geometry} 及 \textsf{fancyhdr} 设定页面排版。本科生页眉的校徽尺寸(高 $1.17$ 厘米)从教务处 Word 模板测量得到。
% \subsubsection{页面尺寸及边距}
% A4 纸双面排版,打印版增加 $0.63$ 厘米装订宽度。本科生 1.5 倍行距,研究生论文双倍行距。
%    \begin{macrocode}
%<*cls>
\def\binding@width{0.63cm}
\def\horizontal@margin{3.17cm}
\def\sht@head@logo@height{1.17cm}
\geometry{
  a4paper,
  twoside, 
}
\ifsht@print
  \geometry{
    inner = \horizontal@margin + \binding@width,
    outer = \horizontal@margin - \binding@width,
  }
\else
  \geometry{
    inner = \horizontal@margin,
    outer = \horizontal@margin,
  }
\fi
\ifsht@undergraduate
  \linespread{1.3}
  \geometry{
    includeheadfoot,
    top        = 1.5cm,
    bottom     = 1.75cm,
    headheight = \sht@head@logo@height,
    headsep    = \baselineskip,
    footskip   = 2.54cm - 1.75cm,
  }
\else
  \linespread{1.6}
  \geometry{
    top        = 2.54cm,
    bottom     = 2.54cm,
    headheight = 12pt,
    headsep    = 17.5pt,
    footskip   = 2.54cm - 1.75cm,
  }
\fi
\raggedbottom
\setlength{\parskip}{0.5ex plus 0.25ex minus 0.25ex}
\setlength{\parindent}{2\ccwd}
%</cls>
%    \end{macrocode}
%
% \subsubsection{页眉页脚}
% \begin{macro}{\sht@head@logo}
% 用于在本科生论文页眉左侧绘制校徽。若校徽文件缺失,则显示缺省报错信息。
%    \begin{macrocode}
%<*cls>
\newcommand\sht@head@logo{%
  \IfFileExists{shanghaitech-emblem.pdf}{%
    \includegraphics[height=\sht@head@logo@height]{shanghaitech-emblem.pdf}%
  }{%
    \fbox{%
      \begin{minipage}[b][\sht@head@logo@height][c]{0.4\columnwidth}%
        \zihao{-5}\bfseries\sffamily\color{ShtRed} \school@logo@missing%
      \end{minipage}%
    }%
  }
}
%</cls>
%    \end{macrocode}
% \end{macro}
%
% 启用 \textsf{fancyhdr} 提供的页面格式:
%    \begin{macrocode}
%<*cls>
\pagestyle{fancy}
%</cls>
%    \end{macrocode}
%
% \begin{macro}{Plain}
% 不显示页眉页脚:
%    \begin{macrocode}
%<*cls>
\fancypagestyle{Plain}{}
%</cls>
%    \end{macrocode}
% \end{macro}
%
% \begin{macro}{RomanNumbered}
% 研究生前言页格式:
%    \begin{macrocode}
%<*cls>
\fancypagestyle{RomanNumbered}{
  \fancyhf{}
  \pagenumbering{Roman}
  \fancyhead[OC]{\footnotesize\nouppercase\leftmark}
  \fancyhead[EC]{\footnotesize\nouppercase\sht@flat@title}
  \fancyfoot[C]{\footnotesize\thepage}
  \renewcommand{\headrulewidth}{0.8pt}
  \renewcommand{\footrulewidth}{0pt}
}
%</cls>
%    \end{macrocode}
% \end{macro}
%
% \begin{macro}{LRNumbered}
% 研究生正文页格式:
%    \begin{macrocode}
%<*cls>
\fancypagestyle{LRNumbered}{
  \fancyhf{}
  \fancyhead[OC]{\footnotesize\nouppercase\leftmark}
  \fancyhead[EC]{\footnotesize\nouppercase\sht@flat@title}
  \fancyfoot[OR]{\footnotesize\thepage}
  \fancyfoot[EL]{\footnotesize\thepage}
  \renewcommand{\headrulewidth}{0.8pt}
  \renewcommand{\footrulewidth}{0pt}
}
%</cls>
%    \end{macrocode}
% \end{macro}
%
% \begin{macro}{LRNumberedAppendix}
% 研究生附录页格式:
%    \begin{macrocode}
%<*cls>
\fancypagestyle{LRNumberedAppendix}{
  \fancyhf{}
  \fancyhead[OC]{\footnotesize 附\hspace*{1\ccwd}录}
  \fancyhead[EC]{\footnotesize\nouppercase\sht@flat@title}
  \fancyfoot[OR]{\footnotesize\thepage}
  \fancyfoot[EL]{\footnotesize\thepage}
  \renewcommand{\headrulewidth}{0.8pt}
  \renewcommand{\footrulewidth}{0pt}
}
%</cls>
%    \end{macrocode}
% \end{macro}
%
% \begin{macro}{RomanNumberedWithLogo}
% 本科生前言页格式:
%    \begin{macrocode}
%<*cls>
\fancypagestyle{RomanNumberedWithLogo}{
  \fancyhf{}
  \pagenumbering{Roman}
  \fancyhead[L]{\sht@head@logo}
  \fancyhead[R]{\zihao{-5}\sffamily\sht@flat@title}
  \fancyfoot[C]{\footnotesize\thepage}
  \renewcommand{\headrulewidth}{0.8pt}
  \renewcommand{\footrulewidth}{0pt}
}
%</cls>
%    \end{macrocode}
% \end{macro}
%
% \begin{macro}{MNNumberedWithLogo}
% 本科生正文页格式。页码格式为“第~m~页\hspace*{1\ccwd}共~n~页”,其中 n 来自 \textsf{pageslts} 宏包提供的 |VeryLastPage|,以确保使用 |\cleardoublepage| 后总页码数仍然正确:
%    \begin{macrocode}
%<*cls>
\fancypagestyle{MNNumberedWithLogo}{
  \fancyhf{}
  \fancyhead[L]{\sht@head@logo}
  \fancyhead[R]{\zihao{-5}\sffamily\sht@flat@title}
  \fancyfoot[C]{%
    \footnotesize%
    第~\thepage~页\hspace*{1\ccwd}共~\lastpageref*{VeryLastPage}~页%
  }
  \renewcommand{\headrulewidth}{0.8pt}
  \renewcommand{\footrulewidth}{0pt}
}
%</cls>
%    \end{macrocode}
% \end{macro}
%
% \subsubsection{文档结构切换开关}
%
% \begin{macro}{\frontmatter}
% 根据本科生/研究生论文格式,重新定义前言页面开关:
%    \begin{macrocode}
%<*cls>
\providecommand{\frontmatter}{}
\LetLtxMacro{\sht@tmp@frontmatter}{\frontmatter}
\renewcommand{\frontmatter}{%
  \sht@tmp@frontmatter%
  \ifsht@undergraduate%
    \pagestyle{RomanNumberedWithLogo}%
  \else%
    \pagestyle{RomanNumbered}%
  \fi%
}
%</cls>
%    \end{macrocode}
% \end{macro}
%
% \begin{macro}{\mainmatter}
% 重新定义正文页面开关,并重设页码计数至 1:
%    \begin{macrocode}
%<*cls>
\providecommand{\mainmatter}{}
\LetLtxMacro{\sht@tmp@mainmatter}{\mainmatter}
\renewcommand{\mainmatter}{%
  \sht@tmp@mainmatter%
  \ifsht@undergraduate%
    \pagestyle{MNNumberedWithLogo}%
  \else%
    \pagestyle{LRNumbered}%
  \fi%
  \pagenumbering{arabic}%
  \setcounter{page}{1}%
}
%</cls>
%    \end{macrocode}
% \end{macro}
%
% 在文档开始时显式设置页码计数,以确保 \textsf{pageslts} 正常工作。
%    \begin{macrocode}
%<*cls>
\AtBeginDocument{%
  \pagenumbering{arabic}%
}
%</cls>
%    \end{macrocode}
%
% \subsection{字体设置}
%
% \shtthesis{} 基本沿用了 \textsf{ctex} 宏包的字体选择方式,并重新实现了 Linux/UNIX 下的字体选择与回退机制。
% \begin{macro}{\sht@fontset}
% \textsf{ctex} 内部选择的字体集,以此判断编译系统。
%    \begin{macrocode}
%<*cls>
\newcommand\sht@fontset{\csname g__ctex_fontset_tl\endcsname}
%</cls>
%    \end{macrocode}
% \end{macro}
%
% \subsubsection{英文字体设置}
% 全平台一致,使用 \TeX{} Gyre 系列字体。
%    \begin{macrocode}
%<*cls>
\setmainfont{texgyretermes} [
  Extension      = .otf,
  UprightFont    = *-regular,
  BoldFont       = *-bold,
  ItalicFont     = *-italic,
  BoldItalicFont = *-bolditalic,
  Ligatures      = TeX,
]
\setsansfont{texgyreheros} [
  Extension      = .otf,
  UprightFont    = *-regular,
  BoldFont       = *-bold,
  ItalicFont     = *-italic,
  BoldItalicFont = *-bolditalic,
  Ligatures      = TeX,
]
\setmonofont{texgyrecursor} [
  Extension      = .otf,
  UprightFont    = *-regular,
  BoldFont       = *-bold,
  ItalicFont     = *-italic,
  BoldItalicFont = *-bolditalic,
  Ligatures      = CommonOff,
]
%</cls>
%    \end{macrocode}
%
% \subsubsection{中文字体设置}
% Windows 环境下使用中易字库和伪粗(auto fakebold),不够优雅但是没办法……
%    \begin{macrocode}
%<*cls>
\ifthenelse{\equal{\sht@fontset}{windows}}{
  \def\fake@bold@factor{2.5}
  \setCJKmainfont{SimSun}[
    AutoFakeBold = \fake@bold@factor,
      ItalicFont = KaiTi,
  ]
  \setCJKsansfont{SimHei}[
    AutoFakeBold = \fake@bold@factor,
  ]
  \setCJKfamilyfont{zhsong}{SimSun}[
    AutoFakeBold = \fake@bold@factor,
  ]
  \setCJKfamilyfont{zhhei}{SimHei}[
    AutoFakeBold = \fake@bold@factor,
  ]
  \setCJKfamilyfont{zhkai}{KaiTi}[
    AutoFakeBold = \fake@bold@factor,
  ]
}{
%</cls>
%    \end{macrocode}
%
% macOS 环境下使用多字重华文字库。注意常规字重楷体必须使用全名“Kaiti SC Regular”,否则 \hologo{XeLaTeX} 引擎可能会找不到字体。
%    \begin{macrocode}
%<*cls>
  \ifthenelse{\equal{\sht@fontset}{mac}}{
    \setCJKmainfont{Songti SC}[
         UprightFont = * Light,
            BoldFont = * Bold,
          ItalicFont = Kaiti SC Regular,
      BoldItalicFont = Kaiti SC Bold,
    ]
    \setCJKsansfont{Heiti SC}[
      UprightFont = * Light,
         BoldFont = * Medium,
    ]
    \setCJKfamilyfont{zhsong}{Songti SC}[
      UprightFont = * Light,
         BoldFont = * Bold,
    ]
    \setCJKfamilyfont{zhhei}{Heiti SC}[
      UprightFont = * Light,
         BoldFont = * Medium,
    ]
    \setCJKfamilyfont{zhkai}{Kaiti SC}[
      UprightFont = * Regular,
         BoldFont = * Bold,
    ]
  }{
%</cls>
%    \end{macrocode}
%
% Linux/UNIX 环境下的中文字体选择逻辑比较繁琐,主要原因是 \hologo{XeTeX} 引擎无法使用字体名称(family name 或 full name)调用 \TeX{} Live 内置字体。目前(\fileversion)的中文字体设置流程为:
% \begin{enumerate}
% \item 检测思源宋体、黑体是否安装,若未检测到则使用 Fandol 宋体、黑体;
%   \begin{enumerate}
%   \item 检测 Light 和 Medium 字重宋体、黑体是否存在,若存在则记录其字体全名;
%   \item 若不存在,则使用 Regular 和 Bold 字重,记录字体全名;
%   \end{enumerate}
% \item 检测方正楷体、仿宋是否安装,若未检测到则使用 Fandol 楷体、仿宋;
% \item 若编译引擎为 \hologo{LuaTeX},则使用包含字重的字体全名(full name)设定字体,完成设置;
% \item 若编译引擎为 \hologo{XeTeX} ,有可能需要使用字体文件名(font file path)调用字体。根据宋体、黑体的字体全名:
%   \begin{enumerate}
%   \item 若使用 Fandol 宋体、黑体,则强制使用 Fandol 楷体、仿宋。按照 Fandol 在 \TeX{} Live 内的文件名设定字体,完成设置;
%   \item 若使用思源宋体、黑体:
%     \begin{enumerate}
%     \item 若使用 Fandol 楷体、仿宋,则按照字体全名指定宋体、黑体,按照文件名指定楷体、仿宋,完成设置;
%     \item 若使用方正楷体、仿宋,则按照字体全名指定所有字体,完成设置;
%     \end{enumerate}
%   \end{enumerate}
% \end{enumerate}
%    \begin{macrocode}
%<*cls>
    \IfFontExistsTF{Noto Serif CJK SC}{
      \def\unix@songti{Noto Serif CJK SC}
    }{
      \IfFontExistsTF{Source Han Serif SC}{
        \def\unix@songti{Source Han Serif SC}
      }{
        \def\unix@songti{FandolSong}
      }
    }
    \IfFontExistsTF{\unix@songti\space Light}{
      \edef\unix@songti@upright{\unix@songti\space Light}
      \edef\unix@songti@bold{\unix@songti\space Medium}
    }{
      \edef\unix@songti@upright{\unix@songti}
      \edef\unix@songti@bold{\unix@songti\space Bold}
    }
    \IfFontExistsTF{Noto Sans CJK SC}{
      \def\unix@heiti{Noto Sans CJK SC}
    }{
      \IfFontExistsTF{Source Han Sans SC}{
        \def\unix@heiti{Source Han Sans SC}
      }{
        \def\unix@heiti{FandolHei}
      }
    }
    \IfFontExistsTF{\unix@heiti\space Medium}{
      \edef\unix@heiti@bold{\unix@heiti\space Medium}
    }{
      \edef\unix@heiti@bold{\unix@heiti\space Bold}
    }
    \IfFontExistsTF{FZKai-Z03}{
      \def\unix@kaiti{FZKai-Z03}
    }{
      \def\unix@kaiti{FandolKai}
    }
    \IfFontExistsTF{FZFangSong-Z02}{
      \def\unix@fangsong{FZFangSong-Z02}
    }{
      \def\unix@fangsong{FandolFang}
    }
    \ifluatex
      \setCJKmainfont{\unix@songti@upright}[
          BoldFont = \unix@songti@bold,
        ItalicFont = \unix@kaiti,
      ]
      \setCJKsansfont{\unix@heiti}[BoldFont=\unix@heiti@bold]
      \setCJKmonofont{\unix@fangsong}
      \setCJKfamilyfont{zhsong}{\unix@songti@upright}[
        BoldFont = \unix@songti@bold,
      ]
      \setCJKfamilyfont{zhhei}{\unix@heiti}[BoldFont=\unix@heiti@bold]
      \setCJKfamilyfont{zhkai}{\unix@kaiti}
      \setCJKfamilyfont{zhfs}{\unix@fangsong}
    \else
      \ifthenelse{\equal{\unix@songti}{FandolSong}}{
        \setCJKmainfont{FandolSong}[
          Extension   = .otf,
          UprightFont = *-Regular,
          BoldFont    = *-Bold,
          ItalicFont  = FandolKai-Regular,
        ]
        \setCJKfamilyfont{zhsong}{FandolSong}[
          Extension   = .otf,
          UprightFont = *-Regular,
          BoldFont    = *-Bold,
        ]
      }{
        \ifthenelse{\equal{\unix@kaiti}{FandolKai}}{
          \setCJKmainfont{\unix@songti@upright}[
                  BoldFont = \unix@songti@bold,
               ItalicFont  = FandolKai-Regular,
            ItalicFeatures = {Extension = .otf},
          ]
        }{
          \setCJKmainfont{\unix@songti@upright}[
              BoldFont = \unix@songti@bold,
            ItalicFont = \unix@kaiti,
          ]
        }
        \setCJKfamilyfont{zhsong}{\unix@songti@upright}[
          BoldFont = \unix@songti@bold,
        ]
      }
      \ifthenelse{\equal{\unix@heiti}{FandolHei}}{
        \setCJKsansfont{FandolHei}[
          Extension   = .otf,
          UprightFont = *-Regular,
          BoldFont    = *-Bold,
        ]
        \setCJKfamilyfont{zhhei}{FandolHei}[
          Extension   = .otf,
          UprightFont = *-Regular,
          BoldFont    = *-Bold,
        ]
      }{
        \setCJKsansfont{\unix@heiti}[BoldFont=\unix@heiti@bold]
        \setCJKfamilyfont{zhhei}{\unix@heiti}[BoldFont=\unix@heiti@bold]
      }
      \ifthenelse{\equal{\unix@kaiti}{FandolKai}}{
        \setCJKfamilyfont{zhkai}{FandolKai}[
          Extension   = .otf,
          UprightFont = *-Regular,
        ]
      }{
        \setCJKfamilyfont{zhkai}{\unix@kaiti}
      }
      \ifthenelse{\equal{\unix@fangsong}{FandolFang}}{
        \setCJKfamilyfont{zhfs}{FandolFang}[
          Extension   = .otf,
          UprightFont = *-Regular,
        ]
        \setCJKmonofont{FandolFang}[
          Extension   = .otf,
          UprightFont = *-Regular,
        ]
      }{
        \setCJKmonofont{\unix@fangsong}
        \setCJKfamilyfont{zhfs}{\unix@fangsong}
      }
    \fi
    \providecommand{\songti}{\CJKfamily{zhsong}}
    \providecommand{\heiti}{\CJKfamily{zhhei}}
    \providecommand{\kaishu}{\CJKfamily{zhkai}}
    \providecommand{\fangsong}{\CJKfamily{zhfs}}
  }
}
%</cls>
%    \end{macrocode}
%
% \subsection{数学环境设置}
% 调整数学环境排版,使其尽可能符合 \href{https://zh.m.wikisource.org/zh-hans/GB_3102.11_物理科学和技术中使用的数学符号}{《GB 3102.11 物理科学和技术中使用的数学符号》}要求。此处参考了 \href{https://github.com/tuna/thuthesis/blob/17acabec987947768e35f0693c678ce654020bc2/thuthesis.dtx#L2452-L2520}{thuthesis} 的实现。
%
% \subsubsection{数学符号}
% 使用斜体(italic)拉丁符号和希腊符号。使用直立(upright)的 $∇$ 和 $∂$,而非其斜体(italic)版本($𝛻$ 和 $𝜕$):
%    \begin{macrocode}
%<*cls>
\unimathsetup{
  math-style = ISO,
  bold-style = ISO,
  nabla      = upright,
  partial    = upright,
}
%</cls>
%    \end{macrocode}
% 使用 $⩽$ 和 $⩾$,而非 $≤$ 和 $≥$:
%    \begin{macrocode}
%<*cls>
\protected\def\le{\leqslant}
\protected\def\ge{\geqslant}
\AtBeginDocument{%
  \renewcommand\leq{\leqslant}%
  \renewcommand\geq{\geqslant}%
}
%</cls>
%    \end{macrocode}
% 使积分上下限置于积分符号正上方和正下方:
%    \begin{macrocode}
%<*cls>
\removenolimits{%
  \int\iint\iiint\iiiint\oint\oiint\oiiint
  \intclockwise\varointclockwise\ointctrclockwise\sumint
  \intbar\intBar\fint\cirfnint\awint\rppolint
  \scpolint\npolint\pointint\sqint\intlarhk\intx
  \intcap\intcup\upint\lowint
}
%</cls>
%    \end{macrocode}
% 使用正体(upright,例如 $\symup{Re}$)而非哥特体(fraktur,例如 $\symfrak{Re}$)表示实部和虚部:
%    \begin{macrocode}
%<*cls>
\AtBeginDocument{%
  \renewcommand{\Re}{\operatorname{Re}}%
  \renewcommand{\Im}{\operatorname{Im}}%
}
%</cls>
%    \end{macrocode}
% 兼容旧式的符号加粗命令 |\bm|、|\boldsymbol| 和方块命令 |\square|,并能够在文本模式中使用 |\checkmark|:
%    \begin{macrocode}
%<*cls>
\newcommand\bm{\symbf}
\renewcommand\boldsymbol{\symbf}
\newcommand\square{\mdlgwhtsquare}
\AtBeginDocument{%
  \renewcommand\checkmark{\ensuremath{✓}}%
}
%</cls>
%    \end{macrocode}
% 允许公式断行:
%    \begin{macrocode}
%<*cls>
\allowdisplaybreaks[4]
%</cls>
%    \end{macrocode}
% 最后,重载研究生论文的公式编号格式,及其 |\eqref| 实现:
%    \begin{macrocode}
%<*cls>
\ifsht@graduate
  \newtagform{dots}{\ldots\ (}{)}
  \usetagform{dots}
  \renewcommand{\eqref}[1]{\textup{(\ref{#1})}}
\fi
%</cls>
%    \end{macrocode}
%
% \subsubsection{数学字体}
% 使用 STIX Two Math 字体,使用直立的积分符号,并能够区分 |\symcal| 和 |\symscr| 两种手写体。关于 stylistic set 的设置和选择,请参考 \textsf{unicode-math} 文档的“Caligraphic vs. Script variants”一节,及 \textsf{stix2-otf} 文档。
%    \begin{macrocode}
%<*cls>
\def\sht@math@font{STIX2Math.otf}
\setmathfont{\sht@math@font}[StylisticSet=8]
\setmathfont{\sht@math@font}[
  range        = {scr, bfscr},
  StylisticSet = 1,
]
%</cls>
%    \end{macrocode}
%
% \subsubsection{编号定理环境}
% 编号模式及排版格式对齐 \textsf{amsthm},并在单独的 group 内定义,以便用户以默认格式自定义其他定理环境。(为什么不直接用 \textsf{amsthm}?因为让 \textsf{amsthm} 匹配中文排版需要很不鲁棒的 dirty hack,譬如 patch command……)
%    \begin{macrocode}
%<*cls>
\begingroup
\theoremstyle{plain}
\theoremseparator{.}
\newtheorem{theorem}{定理}[chapter]
\newtheorem{lemma}{引理}[chapter]
\newtheorem{corollary}{推论}[theorem]
\newtheorem{proposition}{命题}[chapter]
\newtheorem{conjecture}{猜想}[chapter]
\theorembodyfont{\upshape}
\newtheorem{definition}{定义}[chapter]
\newtheorem{axiom}{公理}[chapter]
\newtheorem{example}{例}[chapter]
\newtheorem{exercise}{练习}[chapter]
\newtheorem{problem}{问题}[chapter]
\theoremheaderfont{\kaishu}
\newtheorem{remark}{注}[chapter]
\theoremstyle{nonumberplain}
\theoremheaderfont{\itshape}
\theoremseparator{:\hspace*{-0.5\ccwd}}
\theoremsymbol{\ensuremath{\QED}}
\newtheorem{proof}{证明}
\endgroup
%</cls>
%    \end{macrocode}
%
% \subsection{目录及图表列表}
% 以下代码改编自 \href{https://github.com/mohuangrui/ucasthesis}{ucasthesis},用于生成目录和图表列表,感谢莫晃锐的工作。目前的实现可读性较低,是重构优先级最高的代码。在重构完成前不会加入注释。
%    \begin{macrocode}
%<*cls>
\newcommand{\artxmaincnt}{%
  \ifcsname c@chapter\endcsname%
    chapter%
  \else%
    section%
  \fi%
}
\long\def\artxaux#1{}
\newcommand{\intotocnostar}[3]{%
  #1#2{%
    \phantomsection%
    \addcontentsline{toc}{\expandafter\artxaux\string#1}{#3}%
    #3%
  }
  \markboth{\MakeUppercase{#3}}{}%
}
\newcommand{\intotocstar}[3][\artxmaincnt]{%
  #2%
  \phantomsection%
  \addcontentsline{toc}{#1}{#3}%
  \markboth{\MakeUppercase{#3}}{}%
}
\newcommand{\intotoc}{\@ifstar{\intotocstar}{\intotocnostar}}
\newcommand{\intobmknostar}[4][0]{%
  #2#3{%
    \phantomsection%
    \Hy@writebookmark%
    {}%
    {#4}%
    {\@currentHref}%
    {#1}%
    {toc}%
    #4%
  }%
  \markboth{\MakeUppercase{#4}}{}%
}
\newcommand{\intobmkstar}[3][0]{%
  #2%
  \phantomsection%
  \Hy@writebookmark%
  {}%
  {#3}%
  {\@currentHref}%
  {#1}%
  {toc}%
  \markboth{\MakeUppercase{#3}}{}%
}
\newcommand{\intobmk}{\@ifstar{\intobmkstar}{\intobmknostar}}
\LetLtxMacro{\OriginCleardoublepage}{\cleardoublepage}
\renewcommand{\cleardoublepage}{%
  \ifthenelse{\boolean{sht@undergraduate} \AND \NOT\boolean{sht@print}}{%
    \relax\clearpage%
  }{%
    \OriginCleardoublepage%
  }%
}
\DeclareCaptionFont{wuhaocuti}{\zihao{5}\bfseries}
\captionsetup{
  format = plain,
  hangindent = 2.0em,
  labelsep = quad,
  font = {wuhaocuti},
}
\ifsht@undergraduate
  \captionsetup{
    skip = 6pt,
  }
  \def\sht@toc@chapter@fmt{\zihao{5}\rmfamily}
  \def\sht@toc@section@fmt{\zihao{5}\rmfamily}
\else
  \captionsetup{
    skip = 8pt,
  }
  \def\sht@toc@chapter@fmt{\zihao{4}\heiti}
  \def\sht@toc@section@fmt{\zihao{-4}\heiti}
\fi
\def\@dotsep{1.5mu}
\def\@pnumwidth{2em}
\def\@tocrmarg{2em}
\def\@chaptervspace{1ex}
\renewcommand*{\@dottedtocline}[5]{
  \ifnum #1>\c@tocdepth \else
    \vskip \z@ \@plus.2\p@
    {\leftskip #2\relax \rightskip \@tocrmarg \parfillskip -\rightskip
    \parindent #2\relax\@afterindenttrue
    \interlinepenalty\@M
    \leavevmode \sht@toc@section@fmt
    \@tempdima #3\relax
    \advance\leftskip \@tempdima \null\nobreak\hskip -\leftskip
    {#4}\nobreak
    \leaders\hbox{$\m@th\mkern \@dotsep \cdot\mkern \@dotsep$}\hfill
    \nobreak
    \hb@xt@\@pnumwidth{\hfil\normalfont \normalcolor #5}%
    \par\penalty\@highpenalty}%
  \fi
}
\renewcommand*{\l@part}[2]{
  \ifnum \c@tocdepth >-2\relax
    \addpenalty{-\@highpenalty}%
    \addvspace{2.25em \@plus\p@}%
    \setlength\@tempdima{3em}%
    \begingroup
      \parindent \z@ \rightskip \@pnumwidth
      \parfillskip -\@pnumwidth
      {\leavevmode
      \sht@toc@chapter@fmt #1
      \leaders\hbox{$\m@th\mkern \@dotsep \cdot\mkern \@dotsep$}
      \hfil \hb@xt@\@pnumwidth{\hss #2}}\par
      \nobreak
      \global\@nobreaktrue
      \everypar{\global\@nobreakfalse\everypar{}}%
    \endgroup
  \fi
}
\renewcommand*{\l@chapter}[2]{
  \ifnum \c@tocdepth >\m@ne
    \addpenalty{-\@highpenalty}%
    \ifsht@undergraduate%
    \else%
      \vskip \@chaptervspace \@plus\p@%
    \fi%
    \setlength\@tempdima{1.5em}%
    \begingroup
      \parindent \z@ \rightskip \@pnumwidth
      \parfillskip -\@pnumwidth
      \leavevmode \sht@toc@chapter@fmt
      \advance\leftskip\@tempdima
      \hskip -\leftskip
      #1\nobreak
      \leaders\hbox{$\m@th\mkern \@dotsep \cdot\mkern \@dotsep$}
      \hfil \nobreak\hb@xt@\@pnumwidth{\hss #2}\par
      \penalty\@highpenalty
    \endgroup
  \fi
}
\renewcommand*\l@section{\@dottedtocline{1}{1em}{1.8em}}
\renewcommand*\l@subsection{\@dottedtocline{2}{2em}{2.8em}}
\renewcommand*\l@subsubsection{\@dottedtocline{3}{3em}{3.8em}}
\renewcommand*\l@paragraph{\@dottedtocline{4}{4em}{4.8em}}
\renewcommand*\l@subparagraph{\@dottedtocline{5}{5em}{5.8em}}
\renewcommand*\l@figure{\@dottedtocline{1}{1em}{1.8em}}
\renewcommand*\l@table{\@dottedtocline{1}{1em}{1.8em}}
\setcounter{tocdepth}{2}
\setcounter{secnumdepth}{3}
%</cls>
%    \end{macrocode}
% \begin{macro}{\makeindices}
% 根据学位类型,生成不同格式的目录及图形、表格列表。注意该命令不生成符号列表。
%    \begin{macrocode}
%<*cls>
\newcommand{\makeindices}{%
  \begingroup%
  \linespread{1.2}%
  \hypersetup{linkcolor=black}%
  \intobmk*{\cleardoublepage}{\contentsname}%
  \tableofcontents%
  \ifsht@graduate%
    \intobmk*{\cleardoublepage}{\listfigurename}%
    \listoffigures%
    \intobmk*{\cleardoublepage}{\listtablename}%
    \listoftables%
  \fi%
  \endgroup%
}
%</cls>
%    \end{macrocode}
% \end{macro}
%
% \subsection{章节标题格式}
% 使用 \textsf{ctex} 提供的 |\ctexset| 接口,按照研究生、本科生论文类型设置各级标题格式。
%
% \subsubsection{研究生章节标题}
% 注意学校要求在标题内使用中文黑体+西文 Times New Roman 的组合,所以不能直接用 |\sffamily| 切换字体。这里我们用 |\heiti| 实现上述要求。
%    \begin{macrocode}
%<*cls>
\ctexset {
  chapter = {
    format = \bfseries\heiti\zihao{4}\linespread{1.0}\centering,
    nameformat = {},
    numberformat = \rmfamily,
    titleformat = {},
    name = {第, 章},
    number = \arabic{chapter},
    aftername = \hspace{0.75\ccwd},
    beforeskip = {7pt},
    afterskip = {18pt},
    pagestyle = Plain,
  },
  section = {
    format = \heiti\normalsize\linespread{1.0}\raggedright,
    nameformat = {},
    numberformat = \rmfamily,
    titleformat = {},
    aftername = \hspace{0.75\ccwd},
    beforeskip = {24pt},
    afterskip = {6pt},
  },
  subsection = {
    format = \heiti\normalsize\linespread{1.0}\raggedright,
    nameformat = {},
    numberformat = \rmfamily,
    titleformat = {},
    aftername = \hspace{0.75\ccwd},
    beforeskip = {12pt},
    afterskip = {6pt},
  },
  paragraph = {
    indent = 2\ccwd,
    beforeskip = {0pt},
    afterskip = {0.75\ccwd},
  },
  punct = quanjiao,
  space = auto,
  autoindent = true,
}
%</cls>
%    \end{macrocode}
%
% \subsubsection{本科生章节标题}
% 按照学校要求,本科生论文章节标题需缩进 2 个中文字符。然而一二级标题的字号和正文字号相差过大(一级标题三号字,二级标题四号字,正文五号字),直接按照标题字号缩进会造成严重的视觉动线错位。因此作为标准框架内的补救措施,一二级标题的缩进以 2 个五号字字宽为准。
%
% 在指定 |comfort| 选项后,会减少各级标题和正文间的字号差距(一级标题四号字,二级标题小四号字,正文五号字),且各级标题不再缩进,以提高阅读的舒适性。
%    \begin{macrocode}
%<*cls>
\ifsht@undergraduate
  \newlength{\sht@section@indent}
  \ifsht@comfort
    \def\sht@chapter@fmt{\bfseries\sffamily\zihao{4}\linespread{1.0}\centering}
    \def\sht@section@fmt{\sffamily\zihao{-4}\linespread{1.0}\raggedright}
    \def\sht@subsection@fmt{\sffamily\zihao{5}\linespread{1.0}\raggedright}
    \setlength{\sht@section@indent}{0pt}
  \else
    \def\sht@chapter@fmt{\bfseries\sffamily\zihao{3}\linespread{1.0}\centering}
    \def\sht@section@fmt{\sffamily\zihao{4}\linespread{1.0}\raggedright}
    \def\sht@subsection@fmt{\rmfamily\zihao{-4}\linespread{1.0}\raggedright}
    \setlength{\sht@section@indent}{2\ccwd}
  \fi
  \ctexset{
    chapter = {
      format = \sht@chapter@fmt,
      number = \chinese{chapter},
      numberformat = {},
    },
    section = {
      format = \sht@section@fmt,
      numberformat = {},
      indent = \sht@section@indent,
      beforeskip = {18pt},
    },
    subsection = {
      format = \sht@subsection@fmt,
      numberformat = {},
      indent = \sht@section@indent,
    }
  }
\fi
%</cls>
%    \end{macrocode}
%
% \subsection{其他正文环境}
% 所有列表环境标签缩进 2 个汉字字宽,|description| 环境标签后添加中文冒号。
%    \begin{macrocode}
%<*cls>
\setlist{
  nosep,
  labelindent = 2\ccwd,
  labelwidth = *,
  labelsep = 0.5\ccwd,
  leftmargin = 0em,
  itemindent = *,
  listparindent = 2\ccwd,
}
\newcommand{\textbf@with@colon}[1]{\textbf{#1:}}
\setlist[description]{
  labelsep = 0em,
  format = \normalfont\textbf@with@colon,
}
%</cls>
%    \end{macrocode}
%
% 引用环境为中文楷体+英文意大利体,段首缩进 2 中文字宽,左右边距各增加 2 中文字宽。
%    \begin{macrocode}
%<*cls>
\ExplSyntaxOn
\ctex_patch_cmd:Nnn \quotation { 1.5em } { 2 \ccwd }
\ExplSyntaxOff
\BeforeBeginEnvironment{quotation}{%
  \begingroup%
  \setlength\leftmargini{2\ccwd}%
}
\AtBeginEnvironment{quotation}{%
  \itshape%
}
\AfterEndEnvironment{quotation}{%
  \endgroup%
}
%</cls>
%    \end{macrocode}
%
% \subsection{封面及声明页生成}
% 为了视觉重心的稳定,封面条目的下划线会较引号稍微右移 $0.5$ 个汉字字宽,下划线上的内容会左移 $2.5$ 个汉字字宽。为了方便地定义带多个默认可选参数的 \LaTeX2e 命令,这里直接用了 |expl3| 提供的 |\ProvideDocumentCommand| (错误用法示例,需要重构)。 
%    \begin{macrocode}
%<*cls>
\ExplSyntaxOn
\def\ubox@right@shift{0.5\ccwd}
\def\content@left@shift{2.75\ccwd}
\def\content@left@shift@en{2.5\ccwd}
\def\cover@tab@uline@thick{1.2pt}
\def\cover@tab@entry@width{350pt}
\def\cover@tab@last@entry@width{\cover@tab@entry@width - 2\ccwd - 0.5\ccwd}
\def\cover@tab@entry@width@undergraduate{300pt}
\def\cover@tab@uline@thick@undergraduate{1pt}
%</cls>
%    \end{macrocode}
%
% \begin{macro}{\shifted@uline}
% 在下划线开始前插入 |hspace| 以实现下划线右移。
%    \begin{macrocode}
%<*cls>
\ProvideDocumentCommand{\shifted@uline}{%
    O{\cover@tab@uline@thick} %
    O{\ubox@right@shift} %
    m}{%
  \def\ULthickness{#1}%
  \setlength{\ULdepth}{0.3em}%
  \hspace*{#2}%
  \uline{#3}%
}
%</cls>
%    \end{macrocode}
% \end{macro}
%
% \begin{macro}{\shifted@box}
% 在 box 前插入一个负的 |hspace| 以实现内容左移。
%    \begin{macrocode}
%<*cls>
\ProvideDocumentCommand{\shifted@box}{%
    O{\cover@tab@entry@width} %
    O{\content@left@shift} %
    m}{%
  \makebox[#1][c]{\hspace*{-#2} #3 }%
}
%</cls>
%    \end{macrocode}
% \end{macro}
%
% \begin{macro}{\sht@lines@to@tab}
% 将包含换行符的内容转换为 |tabular| 环境内的多行。
%    \begin{macrocode}
%<*cls>
\ProvideDocumentCommand{\sht@lines@to@tab}{%
    O{\cover@tab@uline@thick} %
    O{\cover@tab@entry@width} %
    O{\content@left@shift} %
    m}{%
  \seq_set_split:NnV \l_tmpa_seq {\\} {#4}
  \clist_set_from_seq:NN \l_tmpa_clist \l_tmpa_seq
  \clist_clear:N \l_tmpb_clist
  \clist_map_inline:Nn \l_tmpa_clist
    {%
      \clist_put_right:Nn \l_tmpb_clist 
        { \shifted@uline[#1]{\shifted@box[#2][#3]{##1}} }
    }
  \clist_use:Nn \l_tmpb_clist { \\ & }
}
\ExplSyntaxOff
%</cls>
%    \end{macrocode}
% \end{macro}
%
% \begin{macro}{\sht@schoollogo}
% 用于在研究生论文封面绘制校徽,校徽缺失时显示报错信息。
%    \begin{macrocode}
%<*cls>
\newcommand\sht@schoollogo{%
  \IfFileExists{shanghaitech-emblem.pdf}{%
    \includegraphics[width=10.48cm]{shanghaitech-emblem.pdf}%
  }{%
    \begin{center}%
      \fbox{%
        \begin{minipage}[t][2.79cm][c]{10.48cm}%
          \centering\bfseries\color{ShtRed} \school@logo@missing%
        \end{minipage}%
      }%
    \end{center}%
  }%
}
%</cls>
%    \end{macrocode}
% \end{macro}
%
% \begin{macro}{\sht@schoollogo@undergraduate}
% 用于在本科生论文封面绘制校徽,校徽缺失时显示报错信息。
%    \begin{macrocode}
%<*cls>
\newcommand\sht@schoollogo@undergraduate{%
  \noindent%
  \IfFileExists{shanghaitech-emblem.pdf}{%
    \includegraphics[width=5.39cm]{shanghaitech-emblem.pdf}%
  }{%
    \fbox{%
      \begin{minipage}[t][1.45cm][c]{0.75\columnwidth}%
        \bfseries\color{ShtRed} \school@logo@missing%
      \end{minipage}%
    }%
  }%
}
%</cls>
%    \end{macrocode}
% \end{macro}
%
% \begin{macro}{\maketitle}
% 根据学位类型及是否匿名,依次生成中英文封面和声明页。
%    \begin{macrocode}
%<*cls>
\renewcommand{\maketitle}{%
  \ifsht@undergraduate
    \sht@maketitle@undergraduate
    \sht@maketitle@undergraduate@en
  \else
    \ifthenelse{\equal{\sht@degree}{doctor}}{
      \sht@maketitle@graduate{博士}
      \sht@maketitle@graduate@en{dissertation}
    }{
      \ifthenelse{\equal{\sht@degree}{master}}{
        \sht@maketitle@graduate{硕士}
        \sht@maketitle@graduate@en{thesis}
      }{
        \sht@error{Degree type `\sht@degree' is not supported yet}
      }
    }
  \fi
  \ifthenelse{\not \boolean{sht@anonymous}}{%
    \ifsht@undergraduate%
      \makedeclarations@undergraduate%
    \else%
      \makedeclarations%
    \fi
  }{}
}
%</cls>
%    \end{macrocode}
% \end{macro}
%
% \subsubsection{研究生中文封面}
%    \begin{macrocode}
%<*cls>
\newcommand{\sht@maketitle@graduate}[1]{%
  \intobmk*{\cleardoublepage}{封面}
  \thispagestyle{empty}
  \begin{center}
    \linespread{1.6}\zihao{4}\bfseries
    \hfill{}
    \ifdefempty{\sht@secret@level}{}{%
      密级:%
      \shifted@uline[1pt][0pt]{%
        \shifted@box[50pt][0pt]{%
          \zihao{5} \sht@secret@level%
        }%
      }%
    }

    \vspace*{\stretch{4}}

    \sht@schoollogo

    \vspace*{\stretch{2}}

    {\zihao{1}\heiti #1学位论文}

    \vspace*{\stretch{3}}

    {\zihao{-3}\heiti \shifted@uline[1.5pt][0pt]{\sht@title}}

    \vspace*{\stretch{3}}

    {
      \songti
      \def\tabcolsep{1pt}
      \def\arraystretch{1.3}
      \begin{tabular}{llc}
        作者姓名:&
          \multicolumn{2}{c}{\shifted@uline{\shifted@box{%
            \ifsht@anonymous%
              \sht@anonymous@str%
            \else%
              \sht@author%
            \fi%
          }}} \\
        指导教师:&
          \multicolumn{2}{c}{\shifted@uline{\shifted@box{%
            \ifsht@anonymous%
              \sht@anonymous@str%
            \else%
              \sht@supervisor%
            \fi%
          }}} \\
        \ifdefempty{\sht@supervisor@institution}{}{ &
          \multicolumn{2}{c}{\shifted@uline{%
            \shifted@box{\sht@supervisor@institution}}} \\
        }
        学位类别:&
          \multicolumn{2}{c}{\shifted@uline{\shifted@box{\sht@degree@name}}} \\
        一级学科:&
          \multicolumn{2}{c}{\shifted@uline{%
            \shifted@box{\sht@discipline@level@i}}} \\
        \multicolumn{2}{l}{学校/学院名称:} &
          \shifted@uline{\shifted@box[\cover@tab@last@entry@width]{%
            \sht@institution}} \\
      \end{tabular}
    }

    \vspace*{\stretch{4}}

    \sht@date

    \vspace{\stretch{4}}
  \end{center}
  \clearpage
  \thispagestyle{empty}
  \cleardoublepage
}
%</cls>
%    \end{macrocode}
%
% \subsubsection{研究生英文封面}
%    \begin{macrocode}
%<*cls>
\newcommand{\sht@maketitle@graduate@en}[1]{
  \intobmk*{\cleardoublepage}{Cover}
  \thispagestyle{empty}
  \begin{center}
    \linespread{1.6}
    \zihao{4}\bfseries

    \vspace*{50pt}

    {\zihao{-3}\bfseries \shifted@uline[1.5pt][0pt]{\sht@title@en}}

    \vspace*{\stretch{3}}

    {
      A~#1~submitted~to\\
      ShanghaiTech~University\\
      in~partial~fulfillment~of~the~requirement\\
      for~the~degree~of\\
      \ifthenelse{\equal{\sht@degree}{doctor}}{
        Doctor~of~Philosophy\\
      }{
        \sht@degree@name@en\\
      }
      in~\sht@discipline@level@i@en

      By

      \ifsht@anonymous%
        \sht@anonymous@str%
      \else%
        \sht@author@en%
      \fi%

      Supervisor:~%
      \ifsht@anonymous%
        \sht@anonymous@str%
      \else%
        \sht@supervisor@en%
      \fi%
    }

    \vspace*{\stretch{3}}

    {\sht@institution@en}

    \vspace*{\stretch{1}}

    {\sht@date@en}

    \vspace{\stretch{3}}
  \end{center}
  \clearpage
  \thispagestyle{empty}
  \cleardoublepage
}
%</cls>
%    \end{macrocode}
%
% \subsubsection{本科生中文封面}
%    \begin{macrocode}
%<*cls>
\newcommand{\sht@maketitle@undergraduate}{%
  \intobmk*{\cleardoublepage}{封面}
  \thispagestyle{empty}
  \sht@schoollogo@undergraduate
  \begin{center}
    \linespread{1.6}

    \vspace*{\stretch{20}}

    {\zihao{2}\bfseries 本科毕业论文(设计)}

    \vspace*{\stretch{54}}

    {
      \zihao{4}
      \def\tabcolsep{1pt}
      \def\arraystretch{1.3}
      \begin{tabular}{cc}
        题\hspace{2\ccwd}目:&
          \sht@lines@to@tab[\cover@tab@uline@thick@undergraduate]%
            [\cover@tab@entry@width@undergraduate]{\sht@title} \\
        学生姓名:&
          \shifted@uline[\cover@tab@uline@thick@undergraduate]%
            {\shifted@box[\cover@tab@entry@width@undergraduate]{%
              \sht@author}} \\
        学\hspace{2\ccwd}号:&
          \shifted@uline[\cover@tab@uline@thick@undergraduate]%
            {\shifted@box[\cover@tab@entry@width@undergraduate]{%
              \sht@author@id}} \\
        入学年份:&
          \shifted@uline[\cover@tab@uline@thick@undergraduate]%
            {\shifted@box[\cover@tab@entry@width@undergraduate]{%
              \sht@entrance@year}} \\
        所在学院:&
          \shifted@uline[\cover@tab@uline@thick@undergraduate]%
            {\shifted@box[\cover@tab@entry@width@undergraduate]{%
              \sht@institution}} \\
        攻读专业:&
          \shifted@uline[\cover@tab@uline@thick@undergraduate]%
            {\shifted@box[\cover@tab@entry@width@undergraduate]{%
              \sht@discipline}} \\
        指导教师:&
          \shifted@uline[\cover@tab@uline@thick@undergraduate]%
            {\shifted@box[\cover@tab@entry@width@undergraduate]{%
              \sht@supervisor}} \\
      \end{tabular}
    }

    \vspace*{\stretch{70}}

    {
      \zihao{-4}
      上海科技大学
      \par\sht@date
    }

    \vspace{\stretch{11}}
  \end{center}
  \clearpage
  \thispagestyle{empty}
  \cleardoublepage
}
%</cls>
%    \end{macrocode}
%
% \subsubsection{本科生英文封面}
%    \begin{macrocode}
%<*cls>
\newcommand{\sht@maketitle@undergraduate@en}{%
  \intobmk*{\cleardoublepage}{Cover}
  \thispagestyle{empty}
  \sht@schoollogo@undergraduate
  \begin{center}
    \linespread{1.6}

    \vspace*{\stretch{20}}

    {\zihao{2}\bfseries\sffamily THESIS}

    \vspace*{\stretch{54}}

    {
      \zihao{4}
      \def\tabcolsep{1pt}
      \def\arraystretch{1.3}
      \begin{tabular}{lc}
        Subject: &
          \sht@lines@to@tab[\cover@tab@uline@thick@undergraduate]%
            [\cover@tab@entry@width@undergraduate][\content@left@shift@en]{%
              \sht@title@en} \\
        Student Name: &
          \shifted@uline[\cover@tab@uline@thick@undergraduate]%
            {\shifted@box[\cover@tab@entry@width@undergraduate]%
              [\content@left@shift@en]{\sht@author@en}} \\
        Student ID: &
          \shifted@uline[\cover@tab@uline@thick@undergraduate]%
            {\shifted@box[\cover@tab@entry@width@undergraduate]%
              [\content@left@shift@en]{\sht@author@id}} \\
        Year of Entrance: &
          \shifted@uline[\cover@tab@uline@thick@undergraduate]%
            {\shifted@box[\cover@tab@entry@width@undergraduate]%
              [\content@left@shift@en]{\sht@entrance@year}} \\
        School: &
          \shifted@uline[\cover@tab@uline@thick@undergraduate]%
            {\shifted@box[\cover@tab@entry@width@undergraduate]%
              [\content@left@shift@en]{\sht@institution@en}} \\
        Major: &
          \shifted@uline[\cover@tab@uline@thick@undergraduate]%
            {\shifted@box[\cover@tab@entry@width@undergraduate]%
              [\content@left@shift@en]{\sht@discipline@en}} \\
        Advisor: &
          \shifted@uline[\cover@tab@uline@thick@undergraduate]%
            {\shifted@box[\cover@tab@entry@width@undergraduate]%
              [\content@left@shift@en]{\sht@supervisor@en}} \\
      \end{tabular}
    }

    \vspace*{\stretch{70}}

    {
      \zihao{-4}
      ShanghaiTech University
      \par Date:~\sht@date@en
    }

    \vspace{\stretch{11}}
  \end{center}
  \clearpage
  \thispagestyle{empty}
  \cleardoublepage
}
%</cls>
%    \end{macrocode}
%
% \subsubsection{研究生声明页}
%    \begin{macrocode}
%<*cls>
\newcommand{\makedeclarations}{%
  \intobmk*{\cleardoublepage}{声明}
  \thispagestyle{empty}
  {
    \linespread{1.6}\zihao{-4}

    \vspace*{2ex}

    \begin{center}
      \zihao{4}\bfseries\sffamily 上海科技大学\\研究生学位论文原创性声明
    \end{center}

    本人郑重声明:所呈交的学位论文是本人在导师的指导下独立进行研究工作所取得的成果。%
    尽我所知,除文中已经注明引用的内容外,本论文不包含任何其他个人或集体已经发表或撰写%
    过的研究成果。对论文所涉及的研究工作做出贡献的其他个人和集体,均已在文中以明确方式%
    标明或致谢。

    \vspace*{3ex}

    {\hfill{}作者签名:\hspace*{14em}}

    {\hfill{}日\hspace*{2\ccwd}期:\hspace*{14em}}

    \vspace*{6ex}

    \begin{center}
      \zihao{4}\bfseries\sffamily 上海科技大学\\学位论文授权使用声明
    \end{center}

    本人完全了解并同意遵守上海科技大学有关保存和使用学位论文的规定,即上海科技大学有权%
    保留送交学位论文的副本,允许该论文被查阅,可以按照学术研究公开原则和保护知识产权%
    的原则公布该论文的全部或部分内容,可以采用影印、缩印或其他复制手段保存、汇编本%
    学位论文。
    
    涉密及延迟公开的学位论文在解密或延迟期后适用本声明。

    \vspace*{3ex}

    {\hfill{}作者签名:\hspace*{10em}导师签名:\hspace*{9em}}

    \hfill{}日\hspace*{2\ccwd}期:%
    \hspace*{10em}日\hspace*{2\ccwd}期:\hspace*{9em}%

    \vspace{3ex}
  }
  \clearpage
  \thispagestyle{empty}
  \cleardoublepage
}
%</cls>
%    \end{macrocode}
%
% \subsubsection{本科生声明页}
%    \begin{macrocode}
%<*cls>
\newcommand{\makedeclarations@undergraduate}{%
  \intobmk*{\cleardoublepage}{声明}
  \thispagestyle{empty}
  {
    \linespread{1.6}\zihao{4}

    \vspace*{\stretch{1}}

    \begin{center}
      \zihao{-2}\bfseries\sffamily 上海科技大学\\毕业论文(设计)学术诚信声明
    \end{center}

    \vspace*{\stretch{1}}

    本人郑重声明:所呈交的毕业论文(设计),是本人在导师的指导下,独立进行研究工作%
    所取得的成果。除文中已经注明引用的内容外,本论文不包含任何其他个人或集体已经发表%
    或撰写过的作品成果。对本文的研究做出重要贡献的个人和集体,均已在文中以明确方式%
    标明。本人完全意识到本声明的法律结果由本人承担。

    \vspace*{\stretch{3}}

    \hfill{}作者签名:\hspace*{9.5\ccwd}

    \vspace*{\stretch{1}}

    \hfill{}日\hspace*{2\ccwd}期:%
    \hspace*{2.5\ccwd}年%
    \hspace*{1.5\ccwd}月%
    \hspace*{1.5\ccwd}日%
    \hspace*{1\ccwd}

    \vspace{\stretch{6}}
  }
  \clearpage
  \thispagestyle{empty}
  \cleardoublepage
  \thispagestyle{empty}
  {
    \linespread{1.6}\zihao{4}

    \vspace*{\stretch{1}}

    \begin{center}
      \zihao{-2}\bfseries\sffamily 上海科技大学\\毕业论文(设计)版权使用授权书
    \end{center}

    \vspace*{\stretch{1}}

    本毕业论文(设计)作者同意学校保留并向国家有关部门或机构送交论文的复印件和电子版,%
    允许论文被查阅和借阅。本人授权上海科技大学可以将本毕业论文(设计)的全部或部分内容%
    编入有关数据库进行检索,可以采用影印、缩印或扫描等复制手段保存和汇编本%
    毕业论文(设计)。

    \hspace*{6\ccwd}\textbf{保\hspace*{1\ccwd}密}$\square$,%
    在\uline{\hspace*{2\ccwd}}年解密后适用本授权书。

    本论文属于

    \hspace*{6\ccwd}\textbf{不保密}$\square$。

    (请在以上方框内打“\checkmark”)

    \vspace*{\stretch{3}}

    \noindent 作者签名:\hspace{12.5\ccwd}指导教师签名:

    \vspace*{\stretch{1}}

    \noindent 日期:%
    \hspace*{2.5\ccwd}年%
    \hspace*{1.5\ccwd}月%
    \hspace*{1.5\ccwd}日%
    \hspace*{6\ccwd}日期:%
    \hspace*{2.5\ccwd}年%
    \hspace*{1.5\ccwd}月%
    \hspace*{1.5\ccwd}日

    \vspace{\stretch{4}}
  }
  \clearpage
  \thispagestyle{empty}
  \cleardoublepage
}
%</cls>
%    \end{macrocode}
%
% \subsection{前言环境}
% 前言中的中英文摘要、符号列表需要在相应环境内录入。
%
% \subsubsection{中文摘要环境}
%    \begin{macrocode}
%<*cls>
\newenvironment{abstract}[1][\sht@null@arg]{%
  \cleardoublepage%
  \ifthenelse{\equal{#1}{flattitle}}{%
    \def\sht@abs@title{\sht@flat@title}%
  }{%
    \def\sht@abs@title{\sht@title}%
  }%
  \ifsht@undergraduate
    \let\clearpage\relax%
    \vspace*{\baselineskip}%
    \begin{center}%
      \zihao{3}\bfseries\sffamily\sht@abs@title%
    \end{center}%
    \vspace*{\baselineskip}%
    \ctexset{chapter/format += \zihao{4}, chapter/beforeskip = 0pt}%
  \fi
  \intobmk\chapter*{摘\hspace{1\ccwd}要}%
}{%
  \vspace{\baselineskip}%
  \ifsht@undergraduate%
    \ifsht@comfort%
      \par\noindent{\bfseries\sffamily 关键词:} \sht@keywords%
    \else%
      \par\noindent{\zihao{-4}\bfseries\sffamily 关键词:} \sht@keywords%
    \fi%
  \else%
    \par\noindent{\bfseries 关键词:} \sht@keywords%
  \fi%
}
%</cls>
%    \end{macrocode}
%
% \subsubsection{英文摘要环境}
%    \begin{macrocode}
%<*cls>
\newenvironment{abstract*}[1][\sht@null@arg]{%
  \cleardoublepage%
  \ifthenelse{\equal{#1}{flattitle}}{%
    \def\sht@abs@title@en{\sht@flat@title@en}%
  }{%
    \def\sht@abs@title@en{\sht@title@en}%
  }%
  \ifsht@undergraduate
    \let\clearpage\relax%
    \vspace*{\baselineskip}%
    \begin{center}%
      \zihao{3}\bfseries\sht@abs@title@en%
    \end{center}%
    \vspace*{\baselineskip}%
    \ctexset{chapter/format += \zihao{4}\rmfamily, chapter/beforeskip = 0pt}%
  \fi
  \intobmk\chapter*{Abstract}%
}{%
  \vspace{\baselineskip}%
  \ifsht@undergraduate%
    \ifsht@comfort%
      \par\noindent{\bfseries Key words:} \sht@keywords@en%
    \else%
      \par\noindent{\zihao{-4}\bfseries Key words:} \sht@keywords@en%
    \fi%
  \else%
    \par\noindent{\bfseries Key Words:} \sht@keywords@en%
  \fi%
}
%</cls>
%    \end{macrocode}
%
% \subsubsection{符号列表环境}
%    \begin{macrocode}
%<*cls>
\newcounter{sht@nomenclature@cnt}
\setcounter{sht@nomenclature@cnt}{0}
\def\sht@null@arg{}
\newenvironment{nomenclatures}[1][\sht@null@arg]{%
  \ifnum\thesht@nomenclature@cnt=0
    \cleardoublepage
    \intobmk\chapter*{符号列表}%
  \fi
  \stepcounter{sht@nomenclature@cnt}
  \ifthenelse{\not \equal{#1}{\sht@null@arg}}{%
    \ifsht@undergraduate%
      \ctexset{section/format += \zihao{-4}, section/indent = 0pt}%
    \fi%
    \section*{#1}%
  }{
    \ifsht@undergraduate%
      \par\vspace{18pt}%
    \else%
      \par\vspace{24pt}%
    \fi%
  }
  \renewcommand{\item}[3][\sht@null@arg]{
    \ifthenelse{\not \equal{##1}{\sht@null@arg}}{
      \par\noindent\makebox[0.15\columnwidth][l]{##2}{{##3}\hfill{##1}}
    }{
      \par\noindent\makebox[0.15\columnwidth][l]{##2}{##3}
    }
  }
  \providecommand{\header}[3][\sht@null@arg]{
    \item[\textbf{##1}]{\textbf{##2}}{\textbf{##3}}
  }
}{
}
%</cls>
%    \end{macrocode}
%
% \subsection{附录及后记环境}
% \begin{macro}{\makebiblio}
% 调整字号、行距并按照规范输出参考文献列表。
%    \begin{macrocode}
%<*cls>
\providecommand{\makebiblio}{%
  \renewcommand{\bibfont}{\zihao{5}}%
  \intotoc*{\cleardoublepage}{\bibname}%
  \urlstyle{same}%
  \printbibliography%
  \urlstyle{tt}%
}
%</cls>
%    \end{macrocode}
% \end{macro}
%
% \begin{macro}{\appendix}
% 除切换至附录模式外,调整目录深度,并根据学位类型设置页眉页脚。
%    \begin{macrocode}
%<*cls>
\LetLtxMacro{\origin@appendix}{\appendix}
\renewcommand{\appendix}{%
  \origin@appendix%
  \intotoc*{\cleardoublepage}{\appendixname}%
  \settocdepth{part}%
  \ifsht@undergraduate%
    \pagestyle{MNNumberedWithLogo}%
  \else%
    \pagestyle{LRNumberedAppendix}%
  \fi%
}
%</cls>
%    \end{macrocode}
% \end{macro}
%
% \begin{macro}{\backmatter}
% 同样地,换至后记模式,并根据学位类型设置页眉页脚。
%    \begin{macrocode}
%<*cls>
\providecommand{\backmatter}{}
\LetLtxMacro{\origin@backmatter}{\backmatter}
\renewcommand{\backmatter}{%
  \origin@backmatter%
  \settocdepth{chapter}%
  \ifsht@undergraduate%
    \pagestyle{MNNumberedWithLogo}%
  \else%
    \pagestyle{LRNumbered}%
  \fi%
  \ifsht@undergraduate%
    \ctexset{section/format += \zihao{-4}, section/indent = 0pt}%
  \fi%
}
%</cls>
%    \end{macrocode}
% \end{macro}
%
% \begin{macro}{\sht@check@resume@title}
% 在撰写个人简历时,第一个被调用的环境会为简历添加章节标题。
%    \begin{macrocode}
%<*cls>
\newcounter{sht@resume@cnt}
\setcounter{sht@resume@cnt}{0}
\newcommand{\sht@check@resume@title}{
  \ifnum\thesht@resume@cnt=0
    \cleardoublepage
    \chapter{作者简历及攻读学位期间发表的学术论文与研究成果}
  \fi
  \stepcounter{sht@resume@cnt}
}
%</cls>
%    \end{macrocode}
% \end{macro}
%
% \begin{environment}{resume}
% 作者个人简历,内容仅在非匿名环境显示。
%    \begin{macrocode}
%<*cls>
\newenvironment{resume}{%
  \sht@check@resume@title
  \section*{作者简历:}
  \ifsht@anonymous
    \sht@anonymous@str\par
    \comment
  \fi
}{%
  \ifsht@anonymous
    \endcomment
  \fi
}
%</cls>
%    \end{macrocode}
% \end{environment}
%
% \begin{environment}{publications}
% \begin{environment}{publications*}
% 已发表(或正式接受)的学术论文,分别在非匿名环境(publications)和匿名环境(publications*)下显示。
%    \begin{macrocode}
%<*cls>
\newenvironment{publications}{%
  \sht@check@resume@title
  \ifsht@anonymous
    \comment
  \fi
  \section*{已发表(或正式接受)的学术论文:}
}{%
  \ifsht@anonymous
    \endcomment
  \fi
}
\newenvironment{publications*}{%
  \sht@check@resume@title
  \ifthenelse{\not \boolean{sht@anonymous}}{
    \comment
  }{}
  \section*{已发表(或正式接受)的学术论文:}
}{%
  \ifthenelse{\not \boolean{sht@anonymous}}{
    \endcomment
  }{}
}
%</cls>
%    \end{macrocode}
% \end{environment}
% \end{environment}
%
% \begin{environment}{patents}
% \begin{environment}{patents*}
% 申请或已获得的专利,分别在非匿名环境(patents)和匿名环境(patents*)下显示。
%    \begin{macrocode}
%<*cls>
\newenvironment{patents}{%
  \sht@check@resume@title
  \ifsht@anonymous
    \comment
  \fi
  \section*{申请或已获得的专利:}
}{%
  \ifsht@anonymous
    \endcomment
  \fi
}
\newenvironment{patents*}{%
  \sht@check@resume@title
  \ifthenelse{\not \boolean{sht@anonymous}}{
    \comment
  }{}
  \section*{申请或已获得的专利:}
}{%
  \ifthenelse{\not \boolean{sht@anonymous}}{
    \endcomment
  }{}
}
%</cls>
%    \end{macrocode}
% \end{environment}
% \end{environment}
%
% \begin{environment}{projects}
% 参加的研究项目及获奖情况,仅在非匿名环境下显示。
%    \begin{macrocode}
%<*cls>
\newenvironment{projects}{%
  \sht@check@resume@title
  \section*{参加的研究项目及获奖情况:}
  \ifsht@anonymous
    \sht@anonymous@str\par
    \comment
  \fi
}{%
  \ifsht@anonymous
    \endcomment
  \fi
}
%</cls>
%    \end{macrocode}
% \end{environment}
%
% \begin{environment}{acknowledgement}
% 致谢,仅在非匿名环境下显示。
%    \begin{macrocode}
%<*cls>
\newenvironment{acknowledgement}{%
  \ifsht@anonymous
    \comment
  \else
    \cleardoublepage
    \chapter[致谢]{致\hspace{1\ccwd}谢}\chaptermark{致\hspace{1\ccwd}谢}%
  \fi
}{%
  \ifsht@anonymous
    \endcomment
  \fi
}
%</cls>
%    \end{macrocode}
% \end{environment}
%
% \subsection{文档类后处理}
% 在排版研究生论文或本科生论文的打印版时,确保论文在偶数页结束。
%    \begin{macrocode}
%<*cls>
\AtEndDocument{
  \cleardoublepage
}
%</cls>
%    \end{macrocode}
%
% \section{文档格式宏包 \shtthesisdoc{} 实现}
%    \begin{macrocode}
%<*doc>
\newcommand\shtthesisdoc{\textup{\sffamily shtthesis-doc}}
\ProvidesPackage{shtthesis-doc}
\RequirePackage[heading=true, fontset=none]{ctex}
\RequirePackage{geometry}
\RequirePackage{xcolor}
\RequirePackage{subcaption}
\RequirePackage{ctable}
\RequirePackage{listings}
\RequirePackage{hologo}
\RequirePackage[normalem]{ulem}
\RequirePackage{mathtools}
\RequirePackage[math-style=literal]{unicode-math}
\RequirePackage[hyperref=manual, style=gb7714-2015]{biblatex}
\RequirePackage{hyperref}
\geometry{
  a4paper,
  includeheadfoot,
  top    = 1.5cm,
  bottom = 1.5cm,
  inner  = 3.17cm,
  outer  = 3.17cm,
}
\raggedbottom
\setmainfont{texgyrepagella}[
  Extension      = .otf,
  UprightFont    = *-regular,
  BoldFont       = *-bold,
  ItalicFont     = *-italic,
  BoldItalicFont = *-bolditalic,
  Ligatures      = TeX,
]
\setsansfont{LibertinusSans}[
  Extension      = .otf,
  UprightFont    = *-Regular,
  BoldFont       = *-Bold,
  ItalicFont     = *-Italic,
  Ligatures      = TeX,
]
\setmonofont{Sarasa Mono Slab SC}[
  UprightFont    = * Light,
  BoldFont       = *,
  ItalicFont     = * Light Italic,
  BoldItalicFont = * Italic,
  Ligatures      = CommonOff,
]
\setCJKmainfont{Noto Serif CJK SC}[
  ItalicFont     = FandolKai-Regular,
  ItalicFeatures = {Extension = .otf},
]
\setCJKsansfont{Noto Sans CJK SC}[
  UprightFont = * Regular,
  BoldFont    = * Medium,
]
\setCJKmonofont{Sarasa Mono Slab SC}[
  UprightFont    = * Light,
  BoldFont       = *,
  ItalicFont     = * Light Italic,
  BoldItalicFont = * Italic,
  Ligatures      = CommonOff,
]
\setCJKfamilyfont{zhsong}{Noto Serif CJK SC}
\setCJKfamilyfont{zhhei}{Noto Sans CJK SC}[
  UprightFont = * Regular,
  BoldFont    = * Medium,
]
\setCJKfamilyfont{zhkai}{FandolKai-Regular.otf}
\setCJKfamilyfont{zhfs}{FandolFang-Regular.otf}
\providecommand{\songti}{\CJKfamily{zhsong}}
\providecommand{\heiti}{\CJKfamily{zhhei}}
\providecommand{\kaishu}{\CJKfamily{zhkai}}
\providecommand{\fangsong}{\CJKfamily{zhfs}}
\setmathfont{texgyrepagella-math.otf}
\newtagform{dots}{\ldots\ (}{)}
\definecolor{ShtRed}{RGB}{146,46,23}
\definecolor{fdu@link}{HTML}{990000}
\definecolor{fdu@url}{HTML}{0000B2}
\definecolor{fdu@cite}{HTML}{007F00}
\newcommand\prompt{\textup{\ttfamily \$}}
\lstdefinestyle{lstStyleBase}{%
  basicstyle=\small\ttfamily,
  aboveskip=\medskipamount,
  belowskip=\medskipamount,
  lineskip=0pt,
  boxpos=c,
  showlines=false,
  extendedchars=true,
  upquote=true,
  tabsize=2,
  showtabs=false,
  showspaces=false,
  showstringspaces=false,
  numbers=none,
  linewidth=\linewidth,
  xleftmargin=4pt,
  xrightmargin=0pt,
  resetmargins=false,
  breaklines=true,
  breakatwhitespace=false,
  breakindent=0pt,
  breakautoindent=true,
  columns=flexible,
  keepspaces=true,
  gobble=2,
  framesep=3pt,
  rulesep=1pt,
  framerule=1pt,
  frame=l,
  rulecolor=\color{ShtRed},
  backgroundcolor=\color{gray!5},
  stringstyle=\color{green!40!black!100},
  keywordstyle=\bfseries\color{blue!50!black},
  commentstyle=\slshape\color{black!60},
  escapeinside={`'},
}
\lstdefinestyle{lstStyleShell}{%
  style=lstStyleBase,
  language=bash}
\lstdefinestyle{lstStyleLaTeX}{%
  style=lstStyleBase,
  language=[LaTeX]TeX}
\lstnewenvironment{latex}{\lstset{style=lstStyleLaTeX}}{}
\lstnewenvironment{shell}{\lstset{style=lstStyleShell}}{}
\hypersetup{
  colorlinks = true,
  linkcolor  = fdu@link,
  urlcolor   = fdu@url,
  citecolor  = fdu@cite,
}
\addbibresource{\jobname.bib}
\BiblatexManualHyperrefOn
%</doc>
%    \end{macrocode}
% \clearpage
% \Finale
%
\endinput